\documentclass[paper=a4, fontsize=11pt]{scrartcl} % A4 paper and 11pt font size

\usepackage[T1]{fontenc} % Use 8-bit encoding that has 256 glyphs
\usepackage{fourier} % Use the Adobe Utopia font for the document - comment this line to return to the LaTeX default
\usepackage[english]{babel} % English language/hyphenation
\usepackage{amsmath,amsfonts,amsthm} % Math packages

\usepackage{lipsum} % Used for inserting dummy 'Lorem ipsum' text into the template
\usepackage{minted}
\usepackage{sectsty} % Allows customizing section commands
\allsectionsfont{\scshape} % Make all sections centered, the default font and small caps

\usepackage{fancyhdr} % Custom headers and footers
\pagestyle{fancyplain} % Makes all pages in the document conform to the custom headers and footers
\fancyhead{} % No page header - if you want one, create it in the same way as the footers below
\fancyfoot[L]{} % Empty left footer
\fancyfoot[C]{} % Empty center footer
\fancyfoot[R]{\thepage} % Page numbering for right footer
\renewcommand{\headrulewidth}{0pt} % Remove header underlines
\renewcommand{\footrulewidth}{0pt} % Remove footer underlines
% \setlength{\headheight}{13.6pt} % Customize the height of the header

\numberwithin{equation}{section} % Number equations within sections (i.e. 1.1, 1.2, 2.1, 2.2 instead of 1, 2, 3, 4)
\numberwithin{figure}{section} % Number figures within sections (i.e. 1.1, 1.2, 2.1, 2.2 instead of 1, 2, 3, 4)
\numberwithin{table}{section} % Number tables within sections (i.e. 1.1, 1.2, 2.1, 2.2 instead of 1, 2, 3, 4)

\setlength\parindent{0pt} % Removes all indentation from paragraphs - comment this line for an assignment with lots of text

%----------------------------------------------------------------------------------------
%	TITLE SECTION
%----------------------------------------------------------------------------------------

\newcommand{\horrule}[1]{\rule{\linewidth}{#1}} % Create horizontal rule command with 1 argument of height

\title{	
\normalfont \normalsize 
\textsc{University of California San Diego} \\ [25pt] % Your university, school and/or department name(s)
\horrule{0.5pt} \\[0.4cm] % Thin top horizontal rule
\huge CSE 202: Design and Analysis of Algorithms Homework 2 \\ % The assignment title
\horrule{2pt} \\[0.5cm] % Thick bottom horizontal rule
}

\usepackage[linesnumbered,lined,commentsnumbered]{algorithm2e}
\author{Sai Bi} % Your name

\date{\normalsize\today} % Today's date or a custom date

\begin{document}

\maketitle % Print the title

\section*{Problem 1: Next greater element}

\subsection*{Algorithm description:}
The next greater element can be computed with a single scan of the array from
left to right. Let $A$ be the array, and $ST$ be a stack. We use $NGE$ to store
the NGE of each element.

For an element $i$, we do the following:
\begin{enumerate}
  \item \label{list:a} If $ST$ is empty, push a pair $(i, A[i])$ to $ST$.
  \item If $ST$ is not empty, let $(k, A[k])$ be the top element of $ST$.
    \begin{enumerate}
      \item If $A[k] \geq A[i]$, then push $(i, A[i])$ to $ST$.
      \item If $A[k] < A[i]$, then $NGE[k] = A[i]$. Pop top element from the
        stack. Go to Step~\ref{list:a}.
    \end{enumerate}
\end{enumerate}
After the scan of the array finished, if there are still elements in the stack
$ST$, pop them from the stack mark their NGE as $-1$.

\subsection*{Pseudocode}
\begin{minted}[frame=lines, framesep=2mm]{cpp}
  Input: A[1...N]
  Output: B[1...N]

  ST = stack();
  for (i = 1; i <= N; i++) {
    while (!ST.empty()) {
      (k, v) = ST.top();
      if (v < A[i]) {
        break;
      } else {
        B[k] = A[i];
        ST.pop();
      }
    }
    ST.push_back((i, A[i]));
  } 

  while (!ST.empty()) {
    (k, v) = ST.top();
    B[k] = -1; 
    ST.pop();
  }

  print B;
  return B;
\end{minted}

\subsection*{Proof of correctness}


\section*{Problem 2: Sorted matrix search}
\subsection*{Algorithm description:}
Let the target element be $T$.
Start from the top right corner of the matrix, if current element is smaller
than $T$, we go to the next element in the same column. If current element is 
larger than $T$, we go to the previous element in the same row. If equal, return
current position. Repeat this until we find the target or we have exceeded the
boundary of the matrix.

\subsection*{Pseudocode}
\begin{minted}[frame=lines, framesep=2mm]{cpp}
  Input: m*n matrix M, target element T
  Output: (x, y) // position of T

  y = 1;
  x = n;
  while (true) {
    if (x < 0 || y < 0 || x > n || y > m) {
      return (-1, -1); // target not found.
    }

    if (M[y][x] > T) {
      x--;
    } else if (M[y][x] < T) {
      y++;
    } else {
      return (x, y);
    }
  }
\end{minted}


\section*{Problem 3: Maximum overlap of two intervals}
\subsection*{Algorithm description:}
Sort all the intervals in ascending order of starting point. Initial $t$ with
the endpoint of the first interval, that is $t = b_1$. Scan the rest of intervals from left to
right, and for current interval $[a_i, b_i], i = 2...n$, we calculate the interval overlapping
as $d_i = t - a_i$, and update $t = \max(b_i, t)$. Return the max value of
$d_i$ as the maximum overlap. 

\subsection*{Pseudocode}
\begin{minted}[frame=lines, framesep=2mm]{cpp}
  Input: [a_i, b_i], i = 1...n
  Output: maximum overlapping d

  d = 0;
  t = b_1;
  for (i = 2; i <= n; i++) {
    d = max(t - a_i, d);
    t = max(b_i, t);
  }
  return d;
\end{minted}

\subsection*{Proof of correctness}

\section*{Problem 4: 132 pattern}
\subsection*{Algorithm description:}
Let $A[1...n]$ be the input array,
\begin{enumerate}
\item Scan from left to right, and use $B$ to track the current minimum element,
that is, the minimum of the elements in the subarray that has been scanned thus 
far.

\item Scan from right to left, and use $C$ to track the predecessor of current
element, that is, the maximum element that is smaller than current element in
the subarray that has been scanned thus far. If no predecessor is found, $C[i]$
is set to $+\infty$. If $2 \leq i \leq n-1$ and $B[i] < C[i] < A[i]$, then we have found a
$132$-pattern, return true. 

\end{enumerate}
If no $132$-pattern is found after the second scan, return false.

\subsection*{Pseudocode}
\begin{minted}[frame=lines, framesep=2mm]{cpp}
  Input: A[1...n]
  Output: patternExists
  
  patternExists = false;
  min = +INF;
  B = [];
  for (int i = 1; i <= n; i++) {
    if (A[i] < min) {
      min = A[i];
    }
    B[i] = min;
  }
  
  bst = createBalancedBST(); // Create balanced binary search tree.
  for (int i = n; i >= 2; i--) {
    if (i > 1 && i < n) {
      pred = bst.findPredecessor(A[i]); // Find predecessor of A[i].
      if (B[i] < pred && pred < A[i]) {
        patternExists = true;
        break;
      }
    }
    bst.insert(A[i]);
  }
  return patternExists;
\end{minted}

\section*{Problem 5: Toeplitz matrices}
\subsection*{1}
The sum of two Toeplitz matrices is also Toeplitz.

\textbf{Proof:}
Let $A = (a_{ij}), B = (b_{ij})$ be two Toeplitz matrices, $C = A + B$, then
\begin{align}
  \begin{split}
    c_{i,j} &= a_{i,j} + b_{i,j} \\
           &= a_{i-1,j-1} + b_{i-1,j-1} \\
           &= c_{i-1,j-1} 
  \end{split}
\end{align}
Therefore $C$ is also Toeplitz.

The product of two Toeplitz matrices is not necessarily Toeplitz. A
counter-example is as following:
\begin{align}
  \begin{split}
    A &= \begin{bmatrix}
      1 & 0 \\
      1 & 1 
    \end{bmatrix} \\
    B &= \begin{bmatrix}
      2 & 1 \\
      0 & 2 
    \end{bmatrix} \\
    C &= AB = \begin{bmatrix}
      2 & 1 \\
      2 & 3 
    \end{bmatrix}
  \end{split}
\end{align}
Obviously $C$ is not Toeplitz.

\subsection*{2}
We could use the first row and first column of the matrix to represent a
Toeplitz matrix. That is,
\begin{equation}
  A \overset{\Delta}{=} [A_{1, 1...n}; A_{1...n, 1}], \text{where $A$ is a
  Toeplitz matrix}
\end{equation}
In this way, $A$ can be reconstructed as following:
\[
\begin{bmatrix}
  A_{1,1} & A_{1, 2} & \dots & A_{1, n} \\
  A_{2,1} & A_{1, 1} & \dots & A_{1, n-1} \\
  \vdots & \vdots  & \ddots & \vdots \\
  A_{n,1} & A_{n-1, 1} & \dots & A_{1,1}
\end{bmatrix}
\]
Therefore we just need to add $2n$ elements when adding two Toeplitz
matrices, and the complexity is $O(n)$.

\subsection*{3}
First we construct another Toeplitz matrix $B$ and $C$ as following:
\begin{align}
  \begin{split}
    B &= [0, A_{n, 1}, A_{n-1, 1}, \dots, A_{2, 1}; 0, A_{1, n}, A_{1, n-1},
    \dots, A_{1, 2}] \\
    C &= \begin{bmatrix}
      A & B \\
      B & A \\
    \end{bmatrix}
  \end{split}
\end{align}

Then we know that $C$ is a \textit{circulant matrix}, which has the property
that it can be diagonalized by the DFT matrix, that is $C = F^{-1}\Lambda F$,
where $F$ is the DFT matrix.

To calculate the product of $A$ and vector $v$, we have 
\begin{align}
  \begin{split}
    Av & = (\mathbf{I}_{n\times n} \mathbf{0}_{n\times n}) C \begin{bmatrix}
      v \\
      \mathbf{0}_{n \times 1}
    \end{bmatrix} \\
    &= (\mathbf{I}_{n\times n} \mathbf{0}_{n\times n}) F^{-1}\Lambda F\begin{bmatrix}
      v \\
      \mathbf{0}_{n \times 1}
    \end{bmatrix} 
  \end{split}
  \label{equ:5-3}
\end{align}

We also know that the product of DFT matrix as well as inverse DFT matrix with a vector can be calculated in 
$O(n\log n)$. Therefore for Equation~\ref{equ:5-3}, the time complexity is
analyzed as following:
\begin{enumerate}
  \item $F *: O(n\log n)$
  \item $\Lambda *: O(n) $
  \item $F^{-1} *: O(n\log n) $
  \item $(\mathbf{I}_{n\times n} \mathbf{0}_{n\times n}) * : O(n) $
\end{enumerate}
Thus the time complexity is $O(n\log n)$.

\subsection*{4}
The product of two Toeplitz matrices can be calculated in $O(n^2)$ time.

Let $A$ and $B$ be two $n\times n$ Toeplitz matrices, and $T = AB$. 
\begin{enumerate}
  \item We calculate the first row and first column of $T$. To calculate the
    first column, we multiply $A$ by first column of $B$, which is $O(n\log n)$.
    To calculate the first row, we multiply the first row of $A$ by $B$, which
    is also $O(n\log n)$.
  \item For remaining entries $T_{i,j}$ in $T$, each of them can be calculated in constant 
    time from $T_{i-1, j-1}$. The proof is as following:
      \begin{align}
        \begin{split}
          T_{i-1, j-1} &= \sum_{k=1}^{n} A_{i-1, k} B_{k,j-1} \\
          T_{i, j} &= \sum_{k=1}^{n} A_{i, k} B_{k, j} \\
                &= A_{i,1}B_{1,j} + \sum_{k=2}^{n} A_{i, k} B_{k, j} \\
                &= A_{i,1}B_{1,j} + \sum_{k=2}^n A_{i-1, k-1} B_{k-1, j-1} \\
                &= A_{i,1}B_{1,j} + \sum_{k=1}^n A_{i-1, k} B_{k-1, k} -
                A_{i-1, n}B_{n,j} \\
                &= T_{i-1, j-1} + A_{i,1}B_{1,j} - A_{i-1, n}B_{n,j} \\
        \end{split}
      \end{align}
    Therefore, $T_{i, j}$ can be calculated in constant time from $T_{i-1,j-1}$.
    Given this, we know that all remaining entries can be calculated in $O(n^2)$
    time.
\end{enumerate}
That is, the time complexity of the proposed algorithm for multiplying two
Toeplitz matrices is $O(n^2)$.
\end{document}
