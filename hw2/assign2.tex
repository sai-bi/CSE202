\documentclass[paper=a4, fontsize=11pt]{scrartcl} % A4 paper and 11pt font size

\usepackage[T1]{fontenc} % Use 8-bit encoding that has 256 glyphs
\usepackage{fourier} % Use the Adobe Utopia font for the document - comment this line to return to the LaTeX default
\usepackage[english]{babel} % English language/hyphenation
\usepackage{amsmath,amsfonts,amsthm} % Math packages

\usepackage{lipsum} % Used for inserting dummy 'Lorem ipsum' text into the template
\usepackage{minted}
\usepackage{sectsty} % Allows customizing section commands
\allsectionsfont{\scshape} % Make all sections centered, the default font and small caps

\usepackage{fancyhdr} % Custom headers and footers
\pagestyle{fancyplain} % Makes all pages in the document conform to the custom headers and footers
\fancyhead{} % No page header - if you want one, create it in the same way as the footers below
\fancyfoot[L]{} % Empty left footer
\fancyfoot[C]{} % Empty center footer
\fancyfoot[R]{\thepage} % Page numbering for right footer
\renewcommand{\headrulewidth}{0pt} % Remove header underlines
\renewcommand{\footrulewidth}{0pt} % Remove footer underlines
% \setlength{\headheight}{13.6pt} % Customize the height of the header

\numberwithin{equation}{section} % Number equations within sections (i.e. 1.1, 1.2, 2.1, 2.2 instead of 1, 2, 3, 4)
\numberwithin{figure}{section} % Number figures within sections (i.e. 1.1, 1.2, 2.1, 2.2 instead of 1, 2, 3, 4)
\numberwithin{table}{section} % Number tables within sections (i.e. 1.1, 1.2, 2.1, 2.2 instead of 1, 2, 3, 4)

\setlength\parindent{0pt} % Removes all indentation from paragraphs - comment this line for an assignment with lots of text

%----------------------------------------------------------------------------------------
%	TITLE SECTION
%----------------------------------------------------------------------------------------

\newcommand{\horrule}[1]{\rule{\linewidth}{#1}} % Create horizontal rule command with 1 argument of height

\title{	
\normalfont \normalsize 
\textsc{University of California San Diego} \\ [25pt] % Your university, school and/or department name(s)
\horrule{0.5pt} \\[0.4cm] % Thin top horizontal rule
\huge CSE202: Design and Analysis of Algorithms Homework 2 \\ % The assignment title
\horrule{2pt} \\[0.5cm] % Thick bottom horizontal rule
}

\usepackage[linesnumbered,lined,commentsnumbered]{algorithm2e}
\author{Sai Bi} % Your name

\date{\normalsize\today} % Today's date or a custom date

\begin{document}

\maketitle % Print the title

\section*{Problem 1: Next greater element}

\subsection*{Algorithm description:}
The next greater element can be computed with a single scan of the array from
left to right. Let $A$ be the array, and $ST$ be a stack. We use $NGE$ to store
the NGE of each element.

For an element $i$, we do the following:
\begin{enumerate}
  \item \label{list:a} If $ST$ is empty, push a pair $(i, A[i])$ to $ST$.
  \item If $ST$ is not empty, let $(k, A[k])$ be the top element of $ST$.
    \begin{enumerate}
      \item If $A[k] \geq A[i]$, then push $(i, A[i])$ to $ST$.
      \item If $A[k] < A[i]$, then $NGE[k] = A[i]$. Pop top element from the
        stack. Go to Step~\ref{list:a}.
    \end{enumerate}
\end{enumerate}
After the scan of the array finished, if there are still elements in the stack
$ST$, pop them from the stack mark their NGE as $-1$.

\subsection*{Pseudocode}
\begin{minted}[frame=lines, framesep=2mm]{cpp}
  Input: A[1...N]
  Output: B[1...N]

  ST = stack();
  for (i = 1; i <= N; i++) {
    while (!ST.empty()) {
      (k, v) = ST.top();
      if (v < A[i]) {
        break;
      } else {
        B[k] = A[i];
        ST.pop();
      }
    }
    ST.push_back((i, A[i]));
  } 

  while (!ST.empty()) {
    (k, v) = ST.top();
    B[k] = -1; 
    ST.pop();
  }

  print B;
  return B;
\end{minted}

\subsection*{Proof of correctness}


\section*{Problem 2: Sorted matrix search}
\subsection*{Algorithm description:}
Let the target element be $T$.
Start from the top right corner of the matrix, if current element is smaller
than $T$, we go to the next element in the same column. If current element is 
larger than $T$, we go to the previous element in the same row. If equal, return
current position. Repeat this until we find the target or we have exceeded the
boundary of the matrix.

\subsection*{Pseudocode}
\begin{minted}[frame=lines, framesep=2mm]{cpp}
  Input: m*n matrix M, target element T
  Output: (x, y) // position of T

  y = 1;
  x = n;
  while (true) {
    if (x < 0 || y < 0 || x > n || y > m) {
      return (-1, -1); // target not found.
    }

    if (M[y][x] > T) {
      x--;
    } else if (M[y][x] < T) {
      y++;
    } else {
      return (x, y);
    }
  }
\end{minted}


\section*{Problem 3: Maximum overlap of two intervals}
\subsection*{Algorithm description:}
Sort all the intervals in ascending order of starting point. Initial $t$ with
the endpoint of the first interval, that is $t = b_1$. Scan the rest of intervals from left to
right, and for current interval $[a_i, b_i], i = 2...n$, we calculate the interval overlapping
as $d_i = t - a_i$, and update $t = \max(b_i, t)$. Return the max value of
$d_i$ as the maximum overlap. 

\subsection*{Pseudocode}
\begin{minted}[frame=lines, framesep=2mm]{cpp}
  Input: [a_i, b_i], i = 1...n
  Output: maximum overlapping d

  d = 0;
  t = b_1;
  for (i = 2; i <= n; i++) {
    d = max(t - a_i, d);
    t = max(b_i, t);
  }
  return d;
\end{minted}

\subsection*{Proof of correctness}

\section*{Problem 4: 132 patterns}
\subsection*{Algorithm description:}

\end{document}
