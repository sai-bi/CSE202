\documentclass{article}
\usepackage{amsmath,amsfonts,amsthm,amssymb}
\usepackage{setspace}
\usepackage{fancyhdr}
\usepackage{lastpage}
\usepackage{extramarks}
\usepackage{chngpage}
\usepackage{soul,color}
\usepackage{graphicx,float,wrapfig}
\usepackage{clrscode}
\usepackage{mathrsfs}
\newcommand{\Class}{CSE 202}
% \newcommand{\ClassInstructor}{Russell Impagliazzo}

% Homework Specific Information. Change it to your own
\newcommand{\Title}{Homework 4}
\newcommand{\StudentName}{Sai Bi}
\newcommand{\StudentClass}{}
\newcommand{\StudentNumber}{}

% In case you need to adjust margins:
\topmargin=-0.45in      %
\evensidemargin=0in     %
\oddsidemargin=0in      %
\textwidth=6.5in        %
\textheight=9.0in       %
\headsep=0.25in         %

% Setup the header and footer
\pagestyle{fancy}                                                       %
\lhead{\Title}  %
\rhead{\firstxmark}                                                     %
\lfoot{\lastxmark}                                                      %
\cfoot{}                                                                %
\rfoot{Page\ \thepage\ of\ \protect\pageref{LastPage}}                          %
\renewcommand\headrulewidth{0.4pt}                                      %
\renewcommand\footrulewidth{0.4pt}                                      %

%%%%%%%%%%%%%%%%%%%%%%%%%%%%%%%%%%%%%%%%%%%%%%%%%%%%%%%%%%%%%
% Some tools
\newcommand{\enterProblemHeader}[1]{\nobreak\extramarks{#1}{#1 continued on next page\ldots}\nobreak%
    \nobreak\extramarks{#1 (continued)}{#1 continued on next page\ldots}\nobreak}%
\newcommand{\exitProblemHeader}[1]{\nobreak\extramarks{#1 (continued)}{#1 continued on next page\ldots}\nobreak%
    \nobreak\extramarks{#1}{}\nobreak}%

\providecommand{\myceil}[1]{\left \lceil #1 \right \rceil }
\newcommand{\homeworkProblemName}{}%
\newcounter{homeworkProblemCounter}%
\newenvironment{homeworkProblem}[1][Problem \arabic{homeworkProblemCounter}]%
{\stepcounter{homeworkProblemCounter}%
    \renewcommand{\homeworkProblemName}{#1}%
    \section*{\homeworkProblemName}%
    \enterProblemHeader{\homeworkProblemName}}%
{\exitProblemHeader{\homeworkProblemName}}%

\newcommand{\homeworkSectionName}{}%
\newlength{\homeworkSectionLabelLength}{}%
\newenvironment{homeworkSection}[1]%
{% We put this space here to make sure we're not connected to the above.
    
    \renewcommand{\homeworkSectionName}{#1}%
    \settowidth{\homeworkSectionLabelLength}{\homeworkSectionName}%
    \addtolength{\homeworkSectionLabelLength}{0.25in}%
    \changetext{}{-\homeworkSectionLabelLength}{}{}{}%
    \subsection*{\homeworkSectionName}%
    \enterProblemHeader{\homeworkProblemName\ [\homeworkSectionName]}}%
{\enterProblemHeader{\homeworkProblemName}%
    
    % We put the blank space above in order to make sure this margin
    % change doesn't happen too soon.
    \changetext{}{+\homeworkSectionLabelLength}{}{}{}}%

\newcommand{\Answer}{\ \\\textbf{Answer:} }
\newcommand{\Acknowledgement}[1]{\ \\{\bf Acknowledgement:} #1}
\newcommand{\Complexity}{\vspace{0.3cm} \noindent\textbf{Time Complexity} \\}
\newcommand{\Proof}{\vspace{0.3cm} \noindent\textbf{Proof} \\}
\newcommand{\Algorithm}{\textbf{Algorithm} \\}
\newcommand{\EndProof} { \hfill$\square$ }
\newcommand\equ[1]{\begin{align}\begin{split} #1 \end{split} \end{align}}
%%%%%%%%%%%%%%%%%%%%%%%%%%%%%%%%%%%%%%%%%%%%%%%%%%%%%%%%%%%%%
\setlength\parindent{0pt}
\setlength{\parskip}{0.2cm}%% 
\usepackage{minted}
%%%%%%%%%%%%%%%%%%%%%%%%%%%%%%%%%%%%%%%%%%%%%%%%%%%%%%%%%%%%%
% Make title
\title{\textmd{\bf \Class: \Title}}
\date{}
\author{\textbf{\StudentName}}
%%%%%%%%%%%%%%%%%%%%%%%%%%%%%%%%%%%%%%%%%%%%%%%%%%%%%%%%%%%%%

\begin{document}
\maketitle \thispagestyle{empty}
\section*{Problem 1: Hamiltonian Path}
\Algorithm
Let $V'$ be a set of vertices statisfying $V' \subset V$. We use $M(V', i, j)$ to denote whether
there is a Hamiltonian path from $v_i \in V'$ to $v_j \in V'$ in the graph that only contains vertices in $V'$ and
edges between vertices in $V'$. For initialization, starting from the set $V'$ that contains only $1$ vertex, we set 
$M(V', i, i), |V'| = \{v_i\}$ to \textit{true}. For others, we set $M(V', i, j)$ to true if and only if there exists
a vertex $v_k \in V' - {v_i}$ and there is an edge between $v_i$ and $v_k$, and there is an
Hamiltonian path from $v_k$ to $v_j$, that is $M(V' - \{v_i\}, k, j)$ is true.
And there is a Hamiltonian path from $v_1$ to $v_n$ if and
only if $M(V, 1, n)$ is true.  

\Proof
We prove the following claim: $M(V', i, j)$ is true if and only if there exists a vertex $v_k \in 
V' - \{v_i\}$ and there is an edge between $v_i$ and $v_k$ and also there is a Hamiltonian path from
$v_k$ to $v_j$, that is $M(V' - \{v_i\}, k, j)$ is true.  

If there is a vertex $v_k$, and there is an edge between $v_i$ and $v_k$, and there is a Hamiltonian
path from $v_k$ to $v_j$. Obviously there is a Hamiltonian path in $V'$, and we can get it by
simply add the edge $(v_i, v_k)$ to the Hamiltonian path from $v_k$ to $v_j$.   

Conversely, if $M(V', i, j)$ is true, that is, there is a Hamiltonian path from $v_i$ to $v_j$ in
$V'$, suppose that in the Hamiltonian path, the vertex that comes after $v_i$ is $v_k$. Obviously
$v_k \in V' - \{v_i\}$, there is an edge between $v_i$ and $v_k$, and the path from $v_k$ to $v_j$
is a Hamiltonian path of set $V' - \{v_i\}$. That is, $M(V' - \{v_i\}, k, j)$ is true.  

In summary the claim is true. By induction we can easily prove the correctness of the proposed
algorithm.

\Complexity
There are in total $2^n$ subsets, so there are $2^n \cdot n \cdot n$ triples like $M(V', i, j)$.
To determine whether $M(V', i, j)$ is true, it takes $O(n)$ time since for each $v_k \in V' -
\{v_i\}$ we need to check whether there is an edge between $v_i$ and $v_k$ and also the value of 
$M(V' - \{v_i\}, k, j)$. Therefore the time complexity of the proposed algorithm is $O(2^n \cdot
n^3)$.

\section*{Problem 2}
\Algorithm
Let $r$ be the root of the tree. For each node $v$ in the tree, let $T_v$ be the subtree rooted at 
that node. Let $MC(v, k)$ be the maximum cut of subtree $T_v$ when the set that contains $v$ has
exactly $k$ nodes. Therefore we are trying the determine the value of $MC(r, \frac{|V|}{2})$.

For a leaf node $v$, obviously we know that $k$ can only be $1$, therefore $MC(v, 1) = 0$ for every
leaf node $v$.

For a non-leaf node $v$, suppose $v$ has only one child $u$. Let $(A, B)$ be a maximum tree cut for
$T_v$ and $A$ is the set that contains $v$. We have two possible cases:
\begin{enumerate}
  \item If $u \in A$, in this case we must have $MC(v, k) = MC(u, k-1)$.
  \item If $u \notin A$, in this case we must have $MC(v, k) = c(v, u) + MC(u, |T_u| - k + 1)$, where $|T_u|$
    is the number of vertices in subtree $T_u$, and $c(v, u)$ is the cost of edge between $v$ and
    $u$. 
\end{enumerate}
Therefore, $MC(v, k)$ is calculated as following:
\begin{align}
MC(v, k) = \max(MC(u, k-1), MC(u, |T_u| - k + 1) + c(v, u))
\end{align}

Now we consider the case where $v$ has two children $u$ and $w$. Since edges only exist between
nodes on the same subtree, so we can calculated the maximum cut for $T_u$ and $T_w$ separately, and
then add them together to get the maximum cut for $T_v$. Here
we need to consider every possible way to divide $k-1$ nodes to $T_u$ and $T_w$. Suppose we
divide $n_u$ nodes to $T_u$, and $n_w = k - 1 - n_u$ nodes to $_w$. Then the maximum cut would be
\begin{align}
  MC(v, k) = & \max_{0 \leq n_u \leq k-1, n_w = k-1 -n_u} \Big( \max(MC(u, n_u), MC(u, |T_u| - n_u 
  ) + c(v, u)) \\
  & + \max(MC(w, n_w), MC(w, |T_u| - n_w) + c(v, w))
  \Big)
\end{align}

Given this induction relationship, we can compute all values of $MC(v, k)$ staring from the leaf
node, and finally output the value of $MC(r, |V| / 2)$ as the maximum cut of $T$.

\Proof
We have prove the correctness of the induction relationship. Considering the properties of dynamic
programming, the correctness of the proposed algorithm is obvious.

\Complexity
To get the value of $MC(v, k)$, it takes $O(k)$ time. The maximum value of $k$ is $|T_v|$, where
$|T_v|$ is the number of nodes in subtree $T_v$. So for each vertex $v$, it takes $O(|T_v|^2)$.
Therefore the total complexity of the proposed algorithm is $O(|V|^3)$.

\section*{Problem 3: Heaviest first}
\subsection*{1}
\Proof
If $v\not\in S$, it means that $v$ is deleted in some iteration when the algorithm selects some node
$v'$ which is a neighbour of $v$, which implies that $v'$ must have a weight that is larger than or
equal to that of $v$. Therefore, the claim holds.

\subsection*{2}
Let $W(S^*)$ be the weight of the optimal independent set $S^*$, and let $W(S)$ be the weight of set
$S$. For each $v \in S^*$, we define $v'$ as following:
\begin{enumerate}
  \item If $v \in S$, $v'= v$.
  \item If $v \not\in S$, $v'$ is an neighbour of $v$ in $S$. And we have $w(v') \geq w(v)$. 
\end{enumerate}

Given this definition, since each $u \in S$ can have at most $4$ neighbours in $S^*$, we have the following relationship:
\begin{align}
  W(S^*) & = \sum_{v \in S^*} w(v) \\ 
  & \leq \sum_{v', v \in S^*} w(v')  \\
  & \leq 4 \sum_{u \in S} w(u) \\
  & = 4W(S)
\end{align}
Therefore, the heaviest-first greedy algorithm returns an independent set of total weight at least 
$\frac{1}{4}$ times the maximum total weight of any independent set in the grid graph $G$.


\section*{Problem 4}

\subsection*{1}
Suppose now we have $n$ bins, it takes $\frac{n(n-1)}{2}$ operations to calculate the sum of each
pair, and for each operation we need to calculate the sum of weight of a pair, and compare it with
$W$, which has time complexity of $\log(2 \max(W, \max \{w_i\}))$ (bit operations).

Hence the time complexity of the algorithm can be calculated as following:
\begin{align}
  \sum_{i = 2}^{n} \frac{i(i-1)}{2} \log(2 \max (W, \max \{w_i\})) \leq (n + \log W + \sum_{i=1}^{n}
  \log w_i)^3
\end{align}
Therefore the
merging heuristic always terminates in time polynomial in the size of the input.

\subsection*{2}
The weight of each item is $5, 6, 3, 2, 1$, and $W = 10$. If we first merge the last three items
together, then it will be $5, 6, 6$, which needs $3$ bins. However, we could use two bins by putting
$5, 3$ together and $6, 2, 1$ together.

\subsection*{3}
Suppose that when the algorithm terminates, there are $N$ bins. Obviously we have for any bin $i$
and $j$, we must have $w_i + w_j > W$. So $\sum_{i=1}^{N} w_i > W * \frac{N}{2}$.

Now assume that we could use less than $\frac{N}{2}$ bins to finish this task. Since the weight in
each bin must be less than or equal to $W$, the total weight of items must be less than or equal to 
$\frac{WN}{2}$, which forms a contradiction. 

Hence the number of bins used in the
packing returned by the heuristic is at most twice the minimum possible number of bins.

\section*{Problem 5: Maximum coverage}
\Proof
The greedy algorithm works as following: in each iteration, we pick the set where the total weight
of uncovered elements is maximal. Repeat this until all $k$ sets are picked. 

Let $OPT$ be the weight of the optimal solution to this problem.
Let $x_i$ be the weight of new elements covered by greed algorithm in the $i$-th set it picks. Also
let $y_i = \sum_{j=1}^{i}x_i$, and $z_i = OPT - y_i$.  

\textbf{Claim 1:} $x_{i+1} \geq \frac{z_i}{k}$.

\textbf{Proof:}
Optimal solution covers a weight of $OPT$ within $k$ iterations. That means, at each iteration,
there should some set $U$ whose weight of uncovered elements is greater than or equal to
$\frac{1}{k}$
of the remaining uncovered elements, that is $\frac{z_i}{k}$. Otherwise it is impossible for the
optimal solution to cover a weight of $OPT$ in $k$ steps. Since the greedy algorithm always chooses
the set where the weight of uncovered elements is maximal, the chosen set at each iteration should
be at least $\frac{1}{k}$ of the remaining uncovered elements. That is, $x_{i+1} \geq \frac{z_i}{k}$.

\textbf{Claim 2:} $z_{i+1} \leq (1 - \frac{1}{k})^{i+1} \cdot OPT$
We prove this by induction. When $i = 0$, obviously it holds.

Assume that it holds for $i = p - 1$, that is $z_{p} \leq (1 - \frac{1}{k})^{p} \cdot OPT$. When $i
= p$, we have
\begin{align}
  z_{p+1} & = z_{p} - x_{p+1} \\
  &\leq z_{p} - \frac{1}{k} z_{p} \\
  &\leq (1 - \frac{1}{k})z_p \\
  & \leq (1 - \frac{1}{k})^{p+1} \cdot OPT
\end{align}
By induction, we know that the claim holds.

Given above claims, we have $z_k \leq (1 - \frac{1}{k}) ^k \cdot OPT \leq \frac{OPT}{e}$. Therefore
we have $y_k = OPT - z_k \geq (1 -  \frac{1}{e}) \cdot OPT$.


% \begin{align}
% E(T_k, q) = & \frac{1}{N_T}\sum_{i=1}^{m}\big[S_{k}(p_i) - T_k(q)\big]^2 + \alpha \cdot \frac{1}{N_S}\sum_{j=1}^{n}\big[\hat{S}_{k}(p_j) - T_k(q)\big]^2  \\
% & + \lambda \sum_{p=1}^{N} \sum_{i=1}^{N} w_i[q] \big(\mathcal{P} _p(T_i)[q] - M_p[q] \big)^2
% \end{align}

% \begin{align}
% \frac{\partial E}{\partial T_k(q)} = 0
% \end{align}

% \begin{align}
% \frac{\partial E}{\partial M_k(q)} = 0
% \end{align}

\end{document}





