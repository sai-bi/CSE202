\documentclass{article}
\usepackage{amsmath,amsfonts,amsthm,amssymb}
\usepackage{setspace}
\usepackage{fancyhdr}
\usepackage{lastpage}
\usepackage{extramarks}
\usepackage{chngpage}
\usepackage{soul,color}
\usepackage{graphicx,float,wrapfig}
\usepackage{clrscode}
\usepackage{mathrsfs}
\newcommand{\Class}{CSE 202}
% \newcommand{\ClassInstructor}{Russell Impagliazzo}

% Homework Specific Information. Change it to your own
\newcommand{\Title}{Homework 3}
\newcommand{\StudentName}{Sai Bi}
\newcommand{\StudentClass}{}
\newcommand{\StudentNumber}{}

% In case you need to adjust margins:
\topmargin=-0.45in      %
\evensidemargin=0in     %
\oddsidemargin=0in      %
\textwidth=6.5in        %
\textheight=9.0in       %
\headsep=0.25in         %

% Setup the header and footer
\pagestyle{fancy}                                                       %
\lhead{\Title}  %
\rhead{\firstxmark}                                                     %
\lfoot{\lastxmark}                                                      %
\cfoot{}                                                                %
\rfoot{Page\ \thepage\ of\ \protect\pageref{LastPage}}                          %
\renewcommand\headrulewidth{0.4pt}                                      %
\renewcommand\footrulewidth{0.4pt}                                      %

%%%%%%%%%%%%%%%%%%%%%%%%%%%%%%%%%%%%%%%%%%%%%%%%%%%%%%%%%%%%%
% Some tools
\newcommand{\enterProblemHeader}[1]{\nobreak\extramarks{#1}{#1 continued on next page\ldots}\nobreak%
    \nobreak\extramarks{#1 (continued)}{#1 continued on next page\ldots}\nobreak}%
\newcommand{\exitProblemHeader}[1]{\nobreak\extramarks{#1 (continued)}{#1 continued on next page\ldots}\nobreak%
    \nobreak\extramarks{#1}{}\nobreak}%

\providecommand{\myceil}[1]{\left \lceil #1 \right \rceil }
\newcommand{\homeworkProblemName}{}%
\newcounter{homeworkProblemCounter}%
\newenvironment{homeworkProblem}[1][Problem \arabic{homeworkProblemCounter}]%
{\stepcounter{homeworkProblemCounter}%
    \renewcommand{\homeworkProblemName}{#1}%
    \section*{\homeworkProblemName}%
    \enterProblemHeader{\homeworkProblemName}}%
{\exitProblemHeader{\homeworkProblemName}}%

\newcommand{\homeworkSectionName}{}%
\newlength{\homeworkSectionLabelLength}{}%
\newenvironment{homeworkSection}[1]%
{% We put this space here to make sure we're not connected to the above.
    
    \renewcommand{\homeworkSectionName}{#1}%
    \settowidth{\homeworkSectionLabelLength}{\homeworkSectionName}%
    \addtolength{\homeworkSectionLabelLength}{0.25in}%
    \changetext{}{-\homeworkSectionLabelLength}{}{}{}%
    \subsection*{\homeworkSectionName}%
    \enterProblemHeader{\homeworkProblemName\ [\homeworkSectionName]}}%
{\enterProblemHeader{\homeworkProblemName}%
    
    % We put the blank space above in order to make sure this margin
    % change doesn't happen too soon.
    \changetext{}{+\homeworkSectionLabelLength}{}{}{}}%

\newcommand{\Answer}{\ \\\textbf{Answer:} }
\newcommand{\Acknowledgement}[1]{\ \\{\bf Acknowledgement:} #1}
\newcommand{\Complexity}{\vspace{0.3cm} \noindent\textbf{Time Complexity} \\}
\newcommand{\Proof}{\vspace{0.3cm} \noindent\textbf{Proof} \\}
\newcommand{\Algorithm}{\textbf{Algorithm} \\}
\newcommand{\EndProof} { \hfill$\square$ }
\newcommand\equ[1]{\begin{align}\begin{split} #1 \end{split} \end{align}}
%%%%%%%%%%%%%%%%%%%%%%%%%%%%%%%%%%%%%%%%%%%%%%%%%%%%%%%%%%%%%
\setlength\parindent{0pt}
\setlength{\parskip}{0.2cm}%% 
\usepackage{minted}
%%%%%%%%%%%%%%%%%%%%%%%%%%%%%%%%%%%%%%%%%%%%%%%%%%%%%%%%%%%%%
% Make title
\title{\textmd{\bf \Class: \Title}}
\date{}
\author{\textbf{\StudentName}}
%%%%%%%%%%%%%%%%%%%%%%%%%%%%%%%%%%%%%%%%%%%%%%%%%%%%%%%%%%%%%

\begin{document}
\maketitle \thispagestyle{empty}
\section*{Problem 1: Hamiltonian Path}
\Algorithm
Let $V'$ be a set of vertices statisfying $V' \subset V$. We use $M(V', i, j)$ to denote whether
there is a Hamiltonian path from $v_i \in V'$ to $v_j \in V'$ in the graph that only contains vertices in $V'$ and
edges between vertices in $V'$. For initialization, starting from the set $V'$ that contains only $1$ vertex, we set 
$M(V', i, i), |V'| = \{v_i\}$ to \textit{true}. For others, we set $M(V', i, j)$ to true if and only if there exists
a vertex $v_k \in V' - {v_i}$ and there is an edge between $v_i$ and $v_k$, and there is an
Hamiltonian path from $v_k$ to $v_j$, that is $M(V' - \{v_i\}, k, j)$ is true.
And there is a Hamiltonian path from $v_1$ to $v_n$ if and
only if $M(V, 1, n)$ is true.  

\Proof
We prove the following claim: $M(V', i, j)$ is true if and only if there exists a vertex $v_k \in 
V' - \{v_i\}$ and there is an edge between $v_i$ and $v_k$ and also there is a Hamiltonian path from
$v_k$ to $v_j$, that is $M(V' - \{v_i\}, k, j)$ is true.  

If there is a vertex $v_k$, and there is an edge between $v_i$ and $v_k$, and there is a Hamiltonian
path from $v_k$ to $v_j$. Obviously there is a Hamiltonian path in $V'$, and we can get it by
simply add the edge $(v_i, v_k)$ to the Hamiltonian path from $v_k$ to $v_j$.   

Conversely, if $M(V', i, j)$ is true, that is, there is a Hamiltonian path from $v_i$ to $v_j$ in
$V'$, suppose that in the Hamiltonian path, the vertex that comes after $v_i$ is $v_k$. Obviously
$v_k \in V' - \{v_i\}$, there is an edge between $v_i$ and $v_k$, and the path from $v_k$ to $v_j$
is a Hamiltonian path of set $V' - \{v_i\}$. That is, $M(V' - \{v_i\}, k, j)$ is true.  

In summary the claim is true. By induction we can easily prove the correctness of the proposed
algorithm.

\Complexity
There are in total $2^n$ subsets, so there are $2^n \cdot n \cdot n$ triples like $M(V', i, j)$.
To determine whether $M(V', i, j)$ is true, it takes $O(n)$ time since for each $v_k \in V' -
\{v_i\}$ we need to check whether there is an edge between $v_i$ and $v_k$ and also the value of 
$M(V' - \{v_i\}, k, j)$. Therefore the time complexity of the proposed algorithm is $O(2^n \cdot
n^3)$.
\end{document}

