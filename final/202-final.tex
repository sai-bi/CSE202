\documentclass{article}
\usepackage{amsmath,amsfonts,amsthm,amssymb}
\usepackage{setspace}
\usepackage{fancyhdr}
\usepackage{lastpage}
\usepackage{extramarks}
\usepackage{chngpage}
\usepackage{soul,color}
\usepackage{graphicx,float,wrapfig}
\usepackage{clrscode}
\usepackage{mathrsfs}
\newcommand{\Class}{CSE 202}
% \newcommand{\ClassInstructor}{Russell Impagliazzo}

% Homework Specific Information. Change it to your own
\newcommand{\Title}{Final Exam}
\newcommand{\StudentName}{Sai Bi}
\newcommand{\StudentClass}{}
\newcommand{\StudentNumber}{}

% In case you need to adjust margins:
\topmargin=-0.45in      %
\evensidemargin=0in     %
\oddsidemargin=0in      %
\textwidth=6.5in        %
\textheight=9.0in       %
\headsep=0.25in         %

% Setup the header and footer
\pagestyle{fancy}                                                       %
\lhead{\Title}  %
\rhead{\firstxmark}                                                     %
\lfoot{\lastxmark}                                                      %
\cfoot{}                                                                %
\rfoot{Page\ \thepage\ of\ \protect\pageref{LastPage}}                          %
\renewcommand\headrulewidth{0.4pt}                                      %
\renewcommand\footrulewidth{0.4pt}                                      %

%%%%%%%%%%%%%%%%%%%%%%%%%%%%%%%%%%%%%%%%%%%%%%%%%%%%%%%%%%%%%
% Some tools
\newcommand{\enterProblemHeader}[1]{\nobreak\extramarks{#1}{#1 continued on next page\ldots}\nobreak%
    \nobreak\extramarks{#1 (continued)}{#1 continued on next page\ldots}\nobreak}%
\newcommand{\exitProblemHeader}[1]{\nobreak\extramarks{#1 (continued)}{#1 continued on next page\ldots}\nobreak%
    \nobreak\extramarks{#1}{}\nobreak}%

\providecommand{\myceil}[1]{\left \lceil #1 \right \rceil }
\newcommand{\homeworkProblemName}{}%
\newcounter{homeworkProblemCounter}%
\newenvironment{homeworkProblem}[1][Problem \arabic{homeworkProblemCounter}]%
{\stepcounter{homeworkProblemCounter}%
    \renewcommand{\homeworkProblemName}{#1}%
    \section*{\homeworkProblemName}%
    \enterProblemHeader{\homeworkProblemName}}%
{\exitProblemHeader{\homeworkProblemName}}%

\newcommand{\homeworkSectionName}{}%
\newlength{\homeworkSectionLabelLength}{}%
\newenvironment{homeworkSection}[1]%
{% We put this space here to make sure we're not connected to the above.
    
    \renewcommand{\homeworkSectionName}{#1}%
    \settowidth{\homeworkSectionLabelLength}{\homeworkSectionName}%
    \addtolength{\homeworkSectionLabelLength}{0.25in}%
    \changetext{}{-\homeworkSectionLabelLength}{}{}{}%
    \subsection*{\homeworkSectionName}%
    \enterProblemHeader{\homeworkProblemName\ [\homeworkSectionName]}}%
{\enterProblemHeader{\homeworkProblemName}%
    
    % We put the blank space above in order to make sure this margin
    % change doesn't happen too soon.
    \changetext{}{+\homeworkSectionLabelLength}{}{}{}}%

\newcommand{\Answer}{\ \\\textbf{Answer:} }
\newcommand{\Acknowledgement}[1]{\ \\{\bf Acknowledgement:} #1}
\newcommand{\Complexity}{\vspace{0.3cm} \noindent\textbf{Time Complexity} \vspace{0.2cm} \\}
\newcommand{\Proof}{\vspace{0.3cm} \noindent\textbf{Proof of Correctness} \vspace{0.2cm} \\}
\newcommand{\Algorithm}{\textbf{Algorithm} \vspace{0.2cm}\\}
\newcommand{\EndProof} { \hfill$\square$ }
\newcommand\equ[1]{\begin{align}\begin{split} #1 \end{split} \end{align}}
%%%%%%%%%%%%%%%%%%%%%%%%%%%%%%%%%%%%%%%%%%%%%%%%%%%%%%%%%%%%%
\setlength\parindent{0pt}
\setlength{\parskip}{0.2cm}%% 
\usepackage{minted}
%%%%%%%%%%%%%%%%%%%%%%%%%%%%%%%%%%%%%%%%%%%%%%%%%%%%%%%%%%%%%
% Make title
\title{\textmd{\bf \Class: \Title}}
\date{}
\author{\textbf{\StudentName}}
%%%%%%%%%%%%%%%%%%%%%%%%%%%%%%%%%%%%%%%%%%%%%%%%%%%%%%%%%%%%%

\begin{document}
\maketitle \thispagestyle{empty}

%\section*{Problem 1: Always Non-negative Path}
%\Algorithm
%The algorithm works as following:
%\begin{enumerate}
%  \item For each vertex $v$, We use $P(v)$ to denote the maximum weight path from source $s$ to $v$.
%  Initially we set $P(s) = 0$, and for all other nodes we set $P$ to $-\infty$. 
%\item  \label{list:1} And we update $M(v)$
%    in a manner similar to \textit{Bellman-Ford} algorithm, that is, we iterate over all edges $e \in E$
%    for $|V|$ times, and for any edge $e = (u, v)$, if $P(u) + w(e) \geq 0$, we update value of $P(v) =
%    \max(P(v), P(u) + w(e))$.
%  \item Apply \textit{Depth-First-Search} (DFS) algorithm to find a positive weight cycle and for
%    all vertex $v$ on the cycle, set their $P(v) = +\infty$.
%  \item Repeat Step~\ref{list:1} to update value of $P$ for each vertex.
%  \item There is a non-negative path from $s$ to $t$ if and only if $P(t) \geq 0$.  
%\end{enumerate}
%
%
%\begin{minted}[frame=lines, framesep=2mm, escapeinside=??, mathescape=true]{cpp}
%function Main()
%Input: Graph G = (V, E), s, t
%Output: bool // whether there is a always non-negative path from s to t
%
%P(v) = ?$-\infty$? for any ?$v \in V, v \neq s$?
%P(s) = 0;
%
%BellmanFord();
%FindCycle();
%BellmanFord();
%
%return P(t) ?$\geq$? 0;
%\end{minted}
%
%
%\begin{minted}[frame=lines, framesep=2mm, escapeinside=??, mathescape=true]{cpp}
%function BellmanFord()
%for i = 1 : |V| {
%  for edge (u, v)  in E {
%    if P(u) + w(u, v) ?$\geq$? 0 {
%      P(v) = max(P(u) + w(u, v), P(v)) 
%    }
%  }
%}
%
%\end{minted}
%
%\begin{minted}[frame=lines, framesep=2mm, escapeinside=??, mathescape=true]{cpp}
%function FindCycle()
%S = new stack();
%S.push((s, 0));
%
%while ~S.empty() {
%  (v, m) = S.pop();
%  mark v as visited
%  for all edges (v, u) in E {
%    if P(v)
%  }
%}
%
%\end{minted}

\section*{Problem 2: Pivoting vertices}
\Algorithm
Let $T = (V, E)$. For a vertex $v \in V$, let $a$ be a ancestor of $v$. Let $T_v$ be the subtree
rooted at $v$. Let $C_v = \{u | u \text{ is a child of } v\}$. Let $w(v)$ be the weight of $v$.

We use $W(v, a)$ to denote the cost of subtree $T_v$ when $v$'s closest pivot ancestor is
$a$. And $W(v, v)$ refers to the cost of subtree $T_v$ when $v$ is a pivot.   
We have the following:
\begin{enumerate}
  \item $W(v, v) = w(v) + \sum_{u \in C_v} \min(W(u, v), W(u, u))$. 
  \item $W(v, a) = w(v) - w(a) + \sum_{u \in C_v} \min( W(u, u), W(u, a))$, where $a$ is an ancestor of
    $v$, and $a$ is a pivot vertex.
\end{enumerate}

For base case, if $v$ is a leaf node, we have $W(v, v) = w(v)$, and $W(v, a) = w(v) - w(a)$. We
could go from leaf node and go to upper level and calculate the value of $W(v, a)$ for each
descendent-ancestor pair $v$ and $a$. 

Finally we return $W(r, r)$ where $r$ is the root node, as the minimum total cost.

To get a optimal assignment of vertices, we make root $r$ a pivot vertex. And then we backtrack
to determine the optimal assignment of other vertices, and assume that $W(u, v) < W(u, u)$ we will
make $u$ a regular vertex, otherwise we will make $u$ a pivot vertex.  
 
\Proof
Following we prove the correctness of the induction relationship. 

\textbf{Claim 1:}
$W(v, v) = w(v) + \sum_{u \in C_v} \min(W(u, v), W(u, u))$. 

\textbf{Proof:}
If $v$ is a pivot vertex, obviously for any $u \in C_v$, if $u$ is a pivot vertex,
then the cost of subtree $T_u$ would be $W(u, u)$. If $u$ is a regular vertex, the cost of subtree
would be $W(u, v)$ because now $v$ is the closest pivot ancestor of $u$. So we have 
$W(v, v) = w(v) + \sum_{u \in C_v} \min(W(u, v), W(u, u))$. 

\textbf{Claim 2:}
 $W(v, a) = w(v) - w(a) + \sum_{u \in C_v} \min( W(u, u), W(u, a))$, where $a$ is the closest ancestor of
 $v$.

\textbf{Proof:}
In this case, obviously $v$ is not a pivot vertex, otherwise it turns into the case of $W(v, v)$.
Then the cost of $v$ would be $w(v) - w(a)$. And for any $u \in C_v$, if $u$ is a regular vertex,
the cost of subtree $T_u$ would be $W(u, a)$ since $a$ is the closest ancestor of $u$ now. And if 
$u$ is a pivot vertex, the cost of subtree $T_u$ would be $W(u, u)$. To minimize the cost, obviously we have  
$W(v, a) = w(v) - w(a) + \sum_{u \in C_v} \min( W(u, u), W(u, a))$.

In addition, we can safely assume that root $r$ is a pivot vertex. Because assigning $r$ to be a
regular vertex will not reduce the cost of $r$ since $r$ has no ancestors, and will also not reduce
cost of any of its descendents. 

Therefore $M(r, r)$ must be the minimum cost.  

\Complexity
For a vertex $a$, to calculate $W(v, a)$, the time complexity is $O(|C_v|)$. Therefore to calculate
$W$ for all  descendent-ancestor pairs, the time complexity is $O(|V|^2)$.


\section*{Problem 3: Electronic message transmission systems}
\Algorithm
We build a graph $G = (V, E)$ as following:
\begin{enumerate}
  \item Add a source node $s$ and a sink node $t$ to $G$.
  \item For each site $i$, we add a node $v_i$ to $G$.
  \item For each pair of sites $i, j, i\neq j$, we add a node $v_{ij}$ to $G$
  \item Add an edge from $s$ to $v_{ij}$ with capacity $s_{ij}$.
  \item Add an edge from $v_i$ to $t$ with capacity $c_i$.
  \item Add an edge from $v_{ij}$ to $v_i$ with capacity $+\infty$.
  \item Add an edge from $v_{ij}$ to $v_j$ with capacity $+\infty$.
\end{enumerate}

Given graph $G$, we apply \textit{Push-relabel} algorithm to find a minimum cut $C = (S, T)$ of $G$,
where $s \in S, t \in T$. Then the capacity of $C$ is the maximum sum of the edge
revenues less the vertex costs. And the site we choose are those in the set $\{v_i | v_i \in S\}$ 

\Proof
\textbf{Claim 1:}: $v_{ij} \in S$ if and only if $v_i \in S$ and $v_j \in S$.

\textbf{Proof:} If $v_{ij} \in S$, assume $v_i \not\in S$, then $v_i \in T$. Since there is an edge
from $v_{ij}$ to $v_i$ with capacity $+\infty$, obviously we know that the capacity of cut $C$ will
be $+\infty$, which is not minimum and contradicts with the fact. Therefore $v_i$ must be in $S$.
Similarly we could prove that $v_j \in S$.

If $v_i \in S, v_j \in S$, to get to $v_i$ and $v_j$ from $s$, we have to go through $v_{ij}$, since
$v_i$ and $v_j$ are only connected to $v_{ij}$. Therefore $v_ij \in S$.

Given Claim 1, we know that the only edges from $S$ to $T$ are either edges from $s$ to $v_{ij} \in
T$, or edges from $v_i \in S$ to $t$. Therefore the capacity of the cut could be calculated as
following:

\begin{align}
  |C| & = \sum_{v_{ij} \in T} c(s, v_{ij}) + \sum_{v_i \in S} c(v_i, t) \\
  & = \sum_{v_{ij} \in V} c(s, v_{ij}) - \sum_{v_{ij} \in S} c(s, v_{ij}) + \sum_{v_i \in S} c(v_i,
  t) 
\end{align}

Since $\sum_{v_{ij} \in V} c(s, v_{ij})$ is fixed, by minimizing $|C|$, we are minimizing 
$-\sum_{v_{ij} \in S} c(s, v_{ij}) + \sum_{v_i \in S} c(v_i,t)$, which is exactly the set of
vertices that achieves maximal sum of the edge
revenues less the vertex costs.
  
\Complexity
To build the graph, the time complexity is $(|V|^2)$. And to find the minimum cut with
\textit{Push-relabel} algorithm, the time complexity is $O(|V|^2|E|)$. Therefore the time complexity
of the proposed algorithm is $O(|V|^4)$.

\section*{Problem 4: Maximizing the benefit of unreachable nodes}
\Algorithm
First we construct a new graph $G' = (V', E')$ as following:
\begin{enumerate}
  \item Initially we create a copy of $G$ and make it $G'$.
  \item Change node $r$ to a source node $s$ in $G'$,  
  \item Add a sink node $t$ to $G'$.
  \item For any edge $e'$ in $G'$ set its capacity equal to $c_e$.
  \item For any node $v \in G', v \neq s, v\neq t$, add an edge from $v$ to $t$ with capacity
    equal to $b_v$.
\end{enumerate}

Given graph $G'$, we calculate a minimum cut $C = (S, T)$ where $s \in S, t \in T$, and then the set
of edges attackers should destroy are $A = \{(u, v) | u \in S, v \in S, (u, v) \in E \}$. 

%\Proof
% We are trying to maximize the benifits of disonnecting vertices less the costs of destorying edges.
% Let $(S, T)$ a cut of graph $G'$, if we destory all edges in  $A = \{(u, v) | u \in S, v \in S, (u,
% v) \in E' \}$, then the cost is $C(S, T)$, which is the capacity of the cut. And the benefit
% is the sum of benifits of vertices disconnected from $s$ ($r$). Therefore we have:
% \begin{align} 
  % & \max\sum_{v \in T} b_v  - C(S, T) \\
  % \Leftrightarrow & \min C(S, T) - \sum_{v \in T} b_v \\
  % \Leftrightarrow & \min C(S, T) + \sum_{v \in S} b_v 
% \end{align}
\Proof
By definition of graph cut, we have:
\begin{align}
  C(S, T) = \sum_{(u, v) \in E', u \in S, v\in T} c(u, v)
\end{align}
Here there are two kinds of edges. The first kind of edges are those exist in $E$, and we denote
them as $E_1$, and the second kind of edges are the edges we add from $v$ to $t$, and we denote them
as $E_2$. Therefore we have:
\begin{align}
  C(S, T) & = \sum_{(u, v) \in E', u \in S, v\in T} c(u, v) \\
  & = \sum_{(u, v) \in E_1} c(u, v) + \sum_{(v, t) \in E_2} b_v \\
  & = \sum_{(u, v) \in E_1} c(u, v) + \sum_{v \in S} b_v \\
  & = \sum_{(u, v) \in E_1} c(u, v) + \sum_{v \in V', v \neq s, v \neq t} b_v  - \sum_{v \in T} b_v \\
\end{align}
Therefore by minimizing $C(S, T)$, since $\sum_{v \in V', v \neq s, v \neq t} b_v$ is fixed,  
we are minimizing $ \sum_{(u, v) \in E_1} c(u, v) - \sum_{v \in T} b_v$, that is, we are maximizing 
$ -\sum_{(u, v) \in E_1} c(u, v) + \sum_{v \in T} b_v$. For any $v \in T$, it is disconnected from
$r = s$. Therefore $\sum_{v \in T} b_v$ are the benefits of vertices disconnected from $s$, while
$\sum_{(u, v) \in E_1} c(u, v)$ is the costs of destroying edges. Hence by destroying edges in 
$E_1$, or more formally destroying edges in  $A = \{(u, v) | u \in S, v\in T, (u, v) \in E\}$, we are
maximizing the benefits of disconnecting vertices less the costs of destroying edges.  

\Complexity
To construct the graph, the time complexity is $O(|V| + |E|)$. And to find the minimum cut in $G'$,
we apply \textit{Push-relabel} algorithm, and the complexity is $O(|V|^2|E|)$. Therefore the time
complexity of the proposed algorithm is $O(|V|^4)$.

\section*{Problem 5: IPO}

\Algorithm
The general idea of the proposed algorithm is that every time we select the project whose required
capital is smaller than or equal to the capital we have and also has largest profits. 

To achieve this, we sort all projects in ascending order of minimum capital requirements, and use a
priority queue to keep the set of projects whose minimum capital requirements is smaller than or
equal to our accumulated capital. For every iteration, we find the projects whose minimum capital
requirements under our accumulated capital and add them to the priority queue, and then we select the
top elements in the priority queue as our next project and delete it from the queue. The pseudocode is as following:

\begin{minted}[frame=lines, framesep=2mm, escapeinside=??, mathescape=true]{cpp}
  Input:
    R[1...n] // minimum capital required for each project;
    P[1.. .n] // profits of each project;
    C_0 // initial capital;
    k // number of projects to select;
  
  Output:
    curr_C // accumulated capital after selecting i-th project;

  function IPO() {
    i = 0;
    sort R in ascending order;
    Q = new priority_queue();  
    last_C = ?$-\infty$?;
    curr_C = C_0;
    for (i = 0; i < k; i++) {
      Find the set of projects S whose minimum capital requirement ?$m$? satisfying 
        ?$last_C < m \leq curr_C$?;      
      
      Add the benefits of projects in S to Q;
      if Q is not empty {
        p = Q.pop(); // select the projects with maximum profits.
        last_C = curr_C;
        curr_C = curr_C + p; 
      } else {
        break;
      }
    }
    return curr_C;
  }
\end{minted}


\Proof
Assume that the solution generated by proposed algorithm is not optimal. Let $T'$ be the optimal
solution.

Assume that the first project selected by proposed algorithm different from $T'$ is $i$. Let $p_i$
be the project we select, and let $p_i'$ be the project selected by $T'$. And let $a_i, b_i$ be
$p_i$'s minimum capital requirement and profits, and $a_i', b_i'$ be $p_i'$'s minimum capital
requirement and profits. There are following cases here:
\begin{enumerate}
  \item $b_i > b_i'$. Since this is the first pair where $T'$ and $T$ is different, that is,
    before this selection, $T$ and $T'$ has same accumulated capital. 
    Obviously we could select $p_i$ instead of $p_i'$ in $T'$ to get a larger
    accumulated capital. 
    And also change $p_i'$ to $p_i$ will only affect the following choices.
 
  \item $b_i < b_i'$. Since proposed algorithm is selecting the project with largest profits, and 
    $T$ and $T'$ has same accumulated capital before this selection, obviously the proposed
    algorithm should select $p_i'$ instead of $p_i$, which forms a contradiction.
  
  \item $b_i = b_i', a_i \neq a_i'$. In this case, selecting $p_i$ or $p_i'$ doesn't affect the
    accumulated capital. And since currently accumulated capital is larger than $a_i$ and $a_i'$, so
    changing between $p_i$ and $p_i'$ will not affect following choices.
\end{enumerate}

Given above analysis, we know that by selecting projects as proposed algorithm, we could get a
solution that is optimal. 


\Complexity
The time complexity of sorting $R$ is $O(n\log n)$. And to find the set of projects whose minimum
capital requirements is smaller than or equal to our accumulated capital, we could apply binary
search, which has a time complexity of $O(k\log n)$ for at most $k$ iterations. To add the projects
that satisfy the requirements $last_C < m \leq curr_C$, each project is added to the queue at most
once, which means that this step has a time complexity of $O(n\log n)$ in total (add an element to
priority queue is $O(\log n)$). And also getting the top elements of priority queue has a time
complexity of $O(k \log n)$ in total.

In summary, the time complexity of the proposed algorithm is $O(n\log n)$.

\section*{Problem 7: Cardinality maximum cut}
\Proof
Let $N(A, B)$ be the number of edges with both ends in either $A$ or $B$. And Let $M(A, B)$ be the
number of edges with one endpoint in $A$, and the other endpoint in $B$. Obviously we have $N(A, B)
+ M(A, B) = |E|$.

\textbf{Claim 1:} $N(A, B) \leq M(A, B)$.

\textbf{Proof:}
We prove this by induction. For initialization, $A = \{v_1\}, B = \{v_2\}$. So $N(A, B) = 0$, and 
$M(A, B) = 1$ when there is an edge between $v_1$ and $v_2$, and $0$ otherwise. Therefore we have
$N(A, B) \leq M(A, B)$.

Assume that the claim holds after adding $n - 1 (n \geq 1)$ vertices. When we add another vertex
$v$. According to the algorithm, without loss of generality, we assume $d(v, A) \geq d(v, B)$, and
add $v$ to $B$. In
this case, we have $M(A, B)_n = M(A, B)_{n-1} + d(v, A)$ and $N(A, B)_n = N(A, B)_{n-1} + d(v, B)$.
Therefore $M(A, B) \geq N(A, B)$ also holds after adding $n$ vertices.

In summary, we have proved that $N(A, B) \leq M(A, B)$.

Let $M(A', B')$ be the maximum cut. Given Claim 1, we have:
\begin{align}
M(A', B') & \leq |E| = N(A, B) + M(A, B)\\
  & \leq 2 M(A, B)
\end{align}

Therefore we prove that the proposed algorithm is a factor $\frac{1}{2}$ approximation.

Following we give an example for which the algorithm
outputs a cut which only contains $\frac{1}{2}+\delta$ fraction of the edges compared to the cardinality of the optimal cut
where $\delta > 0$ approaches zero as the number of vertices goes to infinity. We construct a graph
$G$ as  following:


\begin{enumerate}
  \item Add a set of node $(x_1, x_2, \dots, x_n)$ to $G$.
  \item For each $x_i$ and $x_{i+1}$, add an edge between them.
  \item Add a set of node $(y_1, y_2, \dots, y_n)$ to $G$.
  \item For each $y_i$ and $y_{i+1}$, add an edge between them.
  \item For each pair of $x_i$ and $y_j$ where $ 1 \leq i \leq n, 1 \leq j \leq n$, add an edge
    between them.
\end{enumerate}

Obviously the optimal maximum cut would be $A^* = (x_1, x_2, \dots, x_n)$, $B^* = (y_1, y_2, \dots,
y_n)$, and the cardinality of maximum cut is $C(A^*, B^*) = n^2$.

With the greedy algorithm, if we set $A = \{x_1\}, B = \{x_2\}$ initially, then the cut we get would
be $A = (x_1, x_3, x_5, \dots, y_2, y_4, y_6, \dots)$, $B = (x_2, x_4, x_6, \dots, y_1, y_3, y_5,
\dots)$. The cardinality of this cut is $C(A, B) = 2 \cdot \frac{n^2}{4} + 2 \cdot(n-1) =
\frac{n^2}{2} + 2n -2$. Therefore we have:
\begin{align}
  \frac{C(A, B)}{C(A^*, B^*)} &= \frac{\frac{n^2}{2} + 2n -2}{n^2} \\ 
  &= \frac{1}{2} + \frac{2}{n} - \frac{2}{n^2}
\end{align}

Obviously when $n$ approaches infinity, $\frac{C(A, B)}{C(A^*, B^*)}$ approaches $\frac{1}{2}$,
$\delta$ approaches $0$. The upper bound of $\frac{C(A, B)}{C(A^*, B^*)}$ is $1$, which is got when
$n = 2$.


\section*{Problem 8: Broadcast Times for a Tree}
\Algorithm
Let $M(u, v)$ be the maximum time needed for $u$ to broadcast message to a vertex on the subtree   
$T_v$ rooted at $v$. For leaf node we have $M(u, u) = 0$. We have the
following relationship:
\begin{align}
  M(u, v) = \max_{i \in C(v)} (M(v, i) + w(u, v))
  \label{equ:8-1}
\end{align}
Given Equation~\ref{equ:8-1}, we could start from leaf nodes and go upper until we reach the root
node to calculate the value of $M$. After this iteration, we know that $M(u, v)$ holds the maximum
broadcast time needed to broadcast from $u$ to a node in subtree $T_v$.

Now we conduct the new iteration from root to leaf nodes. For a node $u$ and a child $v$ of $u$, 
we use $Q(v, u)$  to represent the maximum time needed to broadcast from $v$ to the subtree $T -
T_v$ through its parent $u$. Starting from the root node, we calculate its broadcast time $t(u)$ as following:
\begin{align}
  t(u) = \max\big(\max_{i \in C(u)} M(u, i), Q(u, P(u)) \big)
\end{align}

For root node $r$, since it doesn't have a parent node, so we set $t(r) =  \max_{i \in C(r)} M(r, i)$.
After this, we update the value of $Q$ for all children $v$ of $u$ as following and go to the next
level:
\begin{align}
  Q(v, u) = \max(Q(u, P(u))  + w(v, u), \max_{i \in C(u), i \neq v} M(u, i) + w(v, u))  
  \label{equ:8-2}
\end{align}

Since root node has parent we set $Q(v, u) =  \max_{i \in C(u), i \neq v} M(u, i) + w(v, u)$ when
$u$ is the root node.  
In this way we could get the broadcast time $t(v)$ for each vertex $v \in T$.

\Proof
\textbf{Claim 1:} 
For a node $v$ and its parent $u$, $M(u, v)$ is the maximum time needed to broadcast from $u$ to the
nodes in subtree $T_v$.

\textbf{Proof:} 
For base case, when $v$ is a leaf node,  $M(u, v) = w(u, v)$, which obviously holds.

Following we prove the correctness of Equation~\ref{equ:8-1}. To calculate the maximum time needed
to broadcast from $u$ to nodes in $T_v$, we have to go through the edge $(u, v)$ to reach $v$. 
And also for each child $i$ of $v$, the maximum time from $v$ to nodes in $T_i$ is $M(v, i)$. Therefore, the
maximum time can be calculated as: $ M(u, v) = \max_{i \in C(v)} (M(v, i) + w(u, v))$. We have
proved the correctness of induction relationship. By properties of dynamic programming, we know that
the claim holds.

\textbf{Claim 2:}
For a node $v$ and its parent $u$, $Q(v, u)$ is the maximum time needed to broadcast from $v$ to the
nodes in subtree $T - T_v$.

\textbf{Proof:}
For base case, when $u$ is the root node, to reach nodes in $T - T_v$, $v$ has to go to $u$ first and then plus the
time needed to broadcast from $u$ to the subtree rooted at the other child of $u$, that is, 
$Q(v, u) = \max_{i \in C(u), i \neq v} M(u, i) + w(v, u)$. 

Following we prove the correctness of Equation~\ref{equ:8-2}. For a node $v$, to reach other nodes
in $T - T_v$, we need to first reach its parent, which needs $w(v, u)$ time. Then there are two
cases here: the first is that we need to reach nodes in $T - T_u$. In this case, the time needed is
$Q(u, P(u)) + w(u, v)$. The second case is that we need to reach nodes in subtrees rooted at other
children $i$ of $u$ except $v$, and the time needed is $M(u, i) + w(v, u)$. Combining these two
cases, we calculate the maximum time needed in two cases as the maximum time needed to broadcast
from $v$ to $T - T_v$. That is 
  $Q(v, u) = \max(Q(u, P(u))  + w(v, u), \max_{i \in C(u), i \neq v} M(u, i) + w(v, u))$. Therefore
  the claim holds.

\textbf{Claim 3:} $t(u)$ is the broadcast time of $u$.

\textbf{Proof:}
To calculate the maximum time for $u$ to reach other nodes, there are two cases. The first is that
we need to reach nodes in subtree $T_u$, and the maximum time is $\max_{i \in C(u)} M(u, i)$,
because we are enumerating the time needed to access nodes in subtrees rooted at every child of $u$.
The second case is that we need to calculate the time needed to access nodes in $T - T_u$, which is 
$Q(u, P(u))$ by claim 2. $T(u)$ is the maximum value of these two cases. The claim is proved.

Combining Claim 1, 2, 3, we have proved the correctness of proposed algorithm.

\Complexity
In the proposed algorithm we visit each edge and vertex of the tree twice, and each time it takes
constant time to calculate the value of $M$ and $Q$ since $T$ is a binary tree. Therefore the time
complexity of the proposed algorithm is $O(|V|)$, where $|V|$ is the number of vertices.   

\end{document}






