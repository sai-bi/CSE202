\documentclass{article}
\usepackage{amsmath,amsfonts,amsthm,amssymb}
\usepackage{setspace}
\usepackage{fancyhdr}
\usepackage{lastpage}
\usepackage{extramarks}
\usepackage{chngpage}
\usepackage{soul,color}
\usepackage{graphicx,float,wrapfig}
\usepackage{clrscode}
\usepackage{mathrsfs}
\newcommand{\Class}{CSE 202}
% \newcommand{\ClassInstructor}{Russell Impagliazzo}

% Homework Specific Information. Change it to your own
\newcommand{\Title}{Homework 2}
\newcommand{\StudentName}{Sai Bi}
\newcommand{\StudentClass}{}
\newcommand{\StudentNumber}{}

% In case you need to adjust margins:
\topmargin=-0.45in      %
\evensidemargin=0in     %
\oddsidemargin=0in      %
\textwidth=6.5in        %
\textheight=9.0in       %
\headsep=0.25in         %

% Setup the header and footer
\pagestyle{fancy}                                                       %
\lhead{\Title}  %
\rhead{\firstxmark}                                                     %
\lfoot{\lastxmark}                                                      %
\cfoot{}                                                                %
\rfoot{Page\ \thepage\ of\ \protect\pageref{LastPage}}                          %
\renewcommand\headrulewidth{0.4pt}                                      %
\renewcommand\footrulewidth{0.4pt}                                      %

%%%%%%%%%%%%%%%%%%%%%%%%%%%%%%%%%%%%%%%%%%%%%%%%%%%%%%%%%%%%%
% Some tools
\newcommand{\enterProblemHeader}[1]{\nobreak\extramarks{#1}{#1 continued on next page\ldots}\nobreak%
    \nobreak\extramarks{#1 (continued)}{#1 continued on next page\ldots}\nobreak}%
\newcommand{\exitProblemHeader}[1]{\nobreak\extramarks{#1 (continued)}{#1 continued on next page\ldots}\nobreak%
    \nobreak\extramarks{#1}{}\nobreak}%

\providecommand{\myceil}[1]{\left \lceil #1 \right \rceil }
\newcommand{\homeworkProblemName}{}%
\newcounter{homeworkProblemCounter}%
\newenvironment{homeworkProblem}[1][Problem \arabic{homeworkProblemCounter}]%
{\stepcounter{homeworkProblemCounter}%
    \renewcommand{\homeworkProblemName}{#1}%
    \section*{\homeworkProblemName}%
    \enterProblemHeader{\homeworkProblemName}}%
{\exitProblemHeader{\homeworkProblemName}}%

\newcommand{\homeworkSectionName}{}%
\newlength{\homeworkSectionLabelLength}{}%
\newenvironment{homeworkSection}[1]%
{% We put this space here to make sure we're not connected to the above.
    
    \renewcommand{\homeworkSectionName}{#1}%
    \settowidth{\homeworkSectionLabelLength}{\homeworkSectionName}%
    \addtolength{\homeworkSectionLabelLength}{0.25in}%
    \changetext{}{-\homeworkSectionLabelLength}{}{}{}%
    \subsection*{\homeworkSectionName}%
    \enterProblemHeader{\homeworkProblemName\ [\homeworkSectionName]}}%
{\enterProblemHeader{\homeworkProblemName}%
    
    % We put the blank space above in order to make sure this margin
    % change doesn't happen too soon.
    \changetext{}{+\homeworkSectionLabelLength}{}{}{}}%

\newcommand{\Answer}{\ \\\textbf{Answer:} }
\newcommand{\Acknowledgement}[1]{\ \\{\bf Acknowledgement:} #1}
\newcommand{\Complexity}{\vspace{0.3cm} \noindent\textbf{Time Complexity} \\}
\newcommand{\Proof}{\vspace{0.3cm} \noindent\textbf{Proof} \\}
\newcommand{\Algorithm}{\textbf{Algorithm} \\}
\newcommand{\EndProof} { \hfill$\square$ }
\newcommand\equ[1]{\begin{align}\begin{split} #1 \end{split} \end{align}}
%%%%%%%%%%%%%%%%%%%%%%%%%%%%%%%%%%%%%%%%%%%%%%%%%%%%%%%%%%%%%


%%%%%%%%%%%%%%%%%%%%%%%%%%%%%%%%%%%%%%%%%%%%%%%%%%%%%%%%%%%%%
% Make title
\title{\textmd{\bf \Class: \Title}}
\date{}
\author{\textbf{\StudentName}}
%%%%%%%%%%%%%%%%%%%%%%%%%%%%%%%%%%%%%%%%%%%%%%%%%%%%%%%%%%%%%

\begin{document}
\maketitle \thispagestyle{empty}
\section*{Problem 1: Nesting Boxes(CLRS)}
\subsection*{Part 1}
Suppose $\mathbf{x} = (x_1,\dots,x_d)$ nests within $\mathbf{y}=(y_1,\dots,y_d)$, and $\mathbf{y}$ nests
within $\mathbf{z} = (z_1, \dots, z_d)$. Then there exists permutations $\pi_1, \pi_2$ on
$\{1,2,\dots,d\}$ such that
\begin{align}
  \begin{split}
    & x_{\pi_1(1)} < y_1, \dots, x_{\pi_1(d)} < y_d \\
    & y_{\pi_2(1)} < z_1, \dots, y_{\pi_2(d)} < z_d \\
  \end{split}
\end{align}
Given this, we have the following:
\begin{align}
  \begin{split}
    z_1 > y_{\pi_2(1)} > x_{\pi_1(\pi_2(1))}, \cdots, 
    z_d > y_{\pi_2(d)} > x_{\pi_1(\pi_2(d))} \\
  \end{split}
\end{align}
Therefore, we know that $\mathbf{x}$ nests within $\mathbf{z}$, and the nesting relation is
transitive. 

\subsection*{Part 2}
\Algorithm
Let $\mathbf{x}, \mathbf{y}$ be two input $d$-dimensional box. We sort the elements in 
$\mathbf{x}$ and $\mathbf{y}$ in ascending order. Then $\mathbf{x}$ nests within $\mathbf{y}$ if and
only if each dimension in sorted $\mathbf{x}$ is smaller than the corresponding dimension in sorted 
$\mathbf{y}$.

\Proof
If each dimension in sorted $\mathbf{x}$ is smaller than the corresponding dimension in sorted
$\mathbf{y}$, then for each $y_i$, we pick $x_{\pi(i)}$ to be $y_i$'s corresponding element in the
sorted $\mathbf{x}$. Obviously we have such a permutation that makes $\mathbf{x}$ nest within
$\mathbf{y}$.

If $\mathbf{x}$ nests within $\mathbf{y}$, let $x^*$ and $y^*$ be the sorted list. Assume that
$x^*_i >= y^*_i$, then we know that 

\begin{align}
  \begin{split}  
  x^*_k \geq y^*_j, \text{for any } k \geq i, j \leq i  \\
  \end{split}
\end{align}
therefore for each $k \geq i$, $x^*_k$ can only corresponding to $y^*_j, j > i$, which is impossible
because the former has $d - i + 1$ elements while the later has only $d-i$ elements. Therefore,
the assumption is wrong, and we have $x^*_i < y^*_i$ for any $1 \leq i \leq d$.

Given above, we can conclude that the proposed algorithm is correct.

\Complexity
It takes $O(d\log d)$ time to sort the list, and $O(d)$ time to compare the elements in sorted
list, therefore the time complexity of the proposed algorithm is $O(d\log d)$.

\subsection*{Part 3}
\Algorithm
We regard each box as a vertex, and if a box $A$ nests within box $B$, we add an directed edge from $A$
to $B$. In this way we can get a graph. Then the task of finding the longest sequence turns into the
task of finding longest path in the graph.

First we show that there is no circle in the graph. Assume that there is a circle, let $C$ be a vertex in a
circle, then there is a path from $C$ to $C$. Since nesting relation is transitive, we know that $C$
nests within itself, which forms a contradiction. Therefore the graph has no circle in it.

Given that the graph is directed and acyclic, we could use \textit{Bellman Ford} algorithm to find the longest
path in the graph. Return the length of the longest path in the graph as final result.



\end{document}

%%%%%%%%%%%%%%%%%%%%%%%%%%%%%%%%%%%%%%%%%%%%%%%%%%%%%%%%%%%%%
