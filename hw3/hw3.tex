\documentclass{article}
\usepackage{amsmath,amsfonts,amsthm,amssymb}
\usepackage{setspace}
\usepackage{fancyhdr}
\usepackage{lastpage}
\usepackage{extramarks}
\usepackage{chngpage}
\usepackage{soul,color}
\usepackage{graphicx,float,wrapfig}
\usepackage{clrscode}
\usepackage{mathrsfs}
\newcommand{\Class}{CSE 202}
% \newcommand{\ClassInstructor}{Russell Impagliazzo}

% Homework Specific Information. Change it to your own
\newcommand{\Title}{Homework 2}
\newcommand{\StudentName}{Sai Bi}
\newcommand{\StudentClass}{}
\newcommand{\StudentNumber}{}

% In case you need to adjust margins:
\topmargin=-0.45in      %
\evensidemargin=0in     %
\oddsidemargin=0in      %
\textwidth=6.5in        %
\textheight=9.0in       %
\headsep=0.25in         %

% Setup the header and footer
\pagestyle{fancy}                                                       %
\lhead{\Title}  %
\rhead{\firstxmark}                                                     %
\lfoot{\lastxmark}                                                      %
\cfoot{}                                                                %
\rfoot{Page\ \thepage\ of\ \protect\pageref{LastPage}}                          %
\renewcommand\headrulewidth{0.4pt}                                      %
\renewcommand\footrulewidth{0.4pt}                                      %

%%%%%%%%%%%%%%%%%%%%%%%%%%%%%%%%%%%%%%%%%%%%%%%%%%%%%%%%%%%%%
% Some tools
\newcommand{\enterProblemHeader}[1]{\nobreak\extramarks{#1}{#1 continued on next page\ldots}\nobreak%
    \nobreak\extramarks{#1 (continued)}{#1 continued on next page\ldots}\nobreak}%
\newcommand{\exitProblemHeader}[1]{\nobreak\extramarks{#1 (continued)}{#1 continued on next page\ldots}\nobreak%
    \nobreak\extramarks{#1}{}\nobreak}%

\providecommand{\myceil}[1]{\left \lceil #1 \right \rceil }
\newcommand{\homeworkProblemName}{}%
\newcounter{homeworkProblemCounter}%
\newenvironment{homeworkProblem}[1][Problem \arabic{homeworkProblemCounter}]%
{\stepcounter{homeworkProblemCounter}%
    \renewcommand{\homeworkProblemName}{#1}%
    \section*{\homeworkProblemName}%
    \enterProblemHeader{\homeworkProblemName}}%
{\exitProblemHeader{\homeworkProblemName}}%

\newcommand{\homeworkSectionName}{}%
\newlength{\homeworkSectionLabelLength}{}%
\newenvironment{homeworkSection}[1]%
{% We put this space here to make sure we're not connected to the above.
    
    \renewcommand{\homeworkSectionName}{#1}%
    \settowidth{\homeworkSectionLabelLength}{\homeworkSectionName}%
    \addtolength{\homeworkSectionLabelLength}{0.25in}%
    \changetext{}{-\homeworkSectionLabelLength}{}{}{}%
    \subsection*{\homeworkSectionName}%
    \enterProblemHeader{\homeworkProblemName\ [\homeworkSectionName]}}%
{\enterProblemHeader{\homeworkProblemName}%
    
    % We put the blank space above in order to make sure this margin
    % change doesn't happen too soon.
    \changetext{}{+\homeworkSectionLabelLength}{}{}{}}%

\newcommand{\Answer}{\ \\\textbf{Answer:} }
\newcommand{\Acknowledgement}[1]{\ \\{\bf Acknowledgement:} #1}
\newcommand{\Complexity}{\vspace{0.3cm} \noindent\textbf{Time Complexity} \\}
\newcommand{\Proof}{\vspace{0.3cm} \noindent\textbf{Proof} \\}
\newcommand{\Algorithm}{\textbf{Algorithm} \\}
\newcommand{\EndProof} { \hfill$\square$ }
\newcommand\equ[1]{\begin{align}\begin{split} #1 \end{split} \end{align}}
%%%%%%%%%%%%%%%%%%%%%%%%%%%%%%%%%%%%%%%%%%%%%%%%%%%%%%%%%%%%%
\setlength\parindent{0pt}
\setlength{\parskip}{0.2cm}%% 
\usepackage{minted}
%%%%%%%%%%%%%%%%%%%%%%%%%%%%%%%%%%%%%%%%%%%%%%%%%%%%%%%%%%%%%
% Make title
\title{\textmd{\bf \Class: \Title}}
\date{}
\author{\textbf{\StudentName}}
%%%%%%%%%%%%%%%%%%%%%%%%%%%%%%%%%%%%%%%%%%%%%%%%%%%%%%%%%%%%%

\begin{document}
\maketitle \thispagestyle{empty}
\section*{Problem 1: Nesting Boxes(CLRS)}
\subsection*{Part 1}
Suppose $\mathbf{x} = (x_1,\dots,x_d)$ nests within $\mathbf{y}=(y_1,\dots,y_d)$, and $\mathbf{y}$ nests
within $\mathbf{z} = (z_1, \dots, z_d)$. Then there exists permutations $\pi_1, \pi_2$ on
$\{1,2,\dots,d\}$ such that
\begin{align}
  \begin{split}
    & x_{\pi_1(1)} < y_1, \dots, x_{\pi_1(d)} < y_d \\
    & y_{\pi_2(1)} < z_1, \dots, y_{\pi_2(d)} < z_d \\
  \end{split}
\end{align}
Given this, we have the following:
\begin{align}
  \begin{split}
    z_1 > y_{\pi_2(1)} > x_{\pi_1(\pi_2(1))}, \cdots, 
    z_d > y_{\pi_2(d)} > x_{\pi_1(\pi_2(d))} \\
  \end{split}
\end{align}
Therefore, we know that $\mathbf{x}$ nests within $\mathbf{z}$, and the nesting relation is
transitive. 

\subsection*{Part 2}
\Algorithm
Let $\mathbf{x}, \mathbf{y}$ be two input $d$-dimensional box. We sort the elements in 
$\mathbf{x}$ and $\mathbf{y}$ in ascending order. Then $\mathbf{x}$ nests within $\mathbf{y}$ if and
only if each dimension in sorted $\mathbf{x}$ is smaller than the corresponding dimension in sorted 
$\mathbf{y}$.

\Proof
If each dimension in sorted $\mathbf{x}$ is smaller than the corresponding dimension in sorted
$\mathbf{y}$, then for each $y_i$, we pick $x_{\pi(i)}$ to be $y_i$'s corresponding element in the
sorted $\mathbf{x}$. Obviously we have such a permutation that makes $\mathbf{x}$ nest within
$\mathbf{y}$.

If $\mathbf{x}$ nests within $\mathbf{y}$, let $x^*$ and $y^*$ be the sorted list. Assume that
$x^*_i >= y^*_i$, then we know that 

\begin{align}
  \begin{split}  
  x^*_k \geq y^*_j, \text{for any } k \geq i, j \leq i  \\
  \end{split}
\end{align}
therefore for each $k \geq i$, $x^*_k$ can only corresponding to $y^*_j, j > i$, which is impossible
because the former has $d - i + 1$ elements while the later has only $d-i$ elements. Therefore,
the assumption is wrong, and we have $x^*_i < y^*_i$ for any $1 \leq i \leq d$.

Given above, we can conclude that the proposed algorithm is correct.

\Complexity
It takes $O(d\log d)$ time to sort the list, and $O(d)$ time to compare the elements in sorted
list, therefore the time complexity of the proposed algorithm is $O(d\log d)$.

\subsection*{Part 3}
\Algorithm
We regard each box as a vertex, and if a box $A$ nests within box $B$, we add an directed edge from $A$
to $B$. In this way we can get a graph. Then the task of finding the longest sequence turns into the
task of finding longest path in the graph.

First we show that there is no circle in the graph. Assume that there is a circle, let $C$ be a vertex in a
circle, then there is a path from $C$ to $C$. Since nesting relation is transitive, we know that $C$
nests within itself, which forms a contradiction. Therefore the graph has no circle in it.

Given that the graph is directed and acyclic, we could use \textit{Bellman Ford} algorithm to find the longest
path in the graph. Return the length of the longest path in the graph as final result.

\Proof
As mentioned above, the longest sequence of nesting of boxes is equivalent to the longest path in
the constructed graph. Since the graph is directed and acyclic, \textit{Bellman Ford} algorithm is
used to find the longest path. Therefore the correctness of the proposed algorithm is obvious.

\Complexity
To construct the graph, we first sort the elements in each box, which takes $O(nd\log d)$ time, and
afterwards for each pair of boxes we check whether they have nesting relations, which takes
$(dn^2)$. To find the longest path in the graph, \textit{Bellman Ford} algorithm runs in time
$O(V+E)$, where $V$ and $E$ are the number of vertices and edges in graph, and therefore the time
complexity for this step is $O(n^2)$. In conclusion, the time complexity of the proposed algorithm
is $O(dn^2)$.


\section*{Problem 2: Changing road conditions}
\Algorithm
We adpot a variant of \textit{Dijkstra's} algorithm to solve this problem. We regard each site as
a vertex in the graph, and if there is a road between two sites, we add an edge between them.
Let $s$ be the starting node, and $w$ be the destination node.

The algorithm works as following:
\begin{enumerate}
  \item Assign each node $v$ an initial time estimation $T(v)$. For the initial node, set it to $0$,
    and for other nodes, set them to $+\infty$. Also we use $Prev(v)$ to represent the previous node 
    of $v$ in the shortest path from starting node to $v$. $Prev(s)$ is set to \textit{null}.

  \item Set the starting node as \textit{current}. And mark all other nodes as unvisited, and add 
    them to the unvisited node set $Q$. 

  \item \label{repeat1}
    For the current node $c$, for all its neighbours $p$, if 
    \begin{align}
      T(p) > T'(p) = T(c) + \min\{f_e(T(c)), e \text{ are edges between } c \text{ and } p\}
    \end{align}
     , update $T(p)$ to $T'(p)$ and assign $c$ to $Prev(v)$.

    \item When we are done considering all neighbour nodes of current node, we mark current node as visited and remove it from $Q$.

    \item If the destination node $w$ has been visited, return $T(w)$ as the time needed to 
      travel from starting point to destination. Also we construct the path by finding previous
      nodes on the path from $s$ to $w$ with information from $Prev$. 
    \item If $Q$ is empty or $T(v) = +\infty$ for all nodes in $Q$, then there is no path from $s$
      to $w$. Return $+\infty$ as the time.      

    \item Otherwise, pick the unvisited node $v$ with smallest $T(v)$ as current node. Go to Step~\ref{repeat1}.
\end{enumerate}

\Proof
We prove the following hypothesis: for each visited node $v$, $T(v)$ is the shortest time needed from $s$ to
$v$. 

We prove this by induction.
For base case, there is only one visited node, that is the starting node $s$. Obviously the
hypothesis is true.

Assume that the hypothesis is true for $n-1$ visited nodes. According to the algorithm, when we
choose the next node to visit, we select the one with minimum $T$. 
Let $v$ be the next node, and $u = Prev(v)$. Then $T(v)$ must be the shortest time needed from 
$s$ to $v$. 
Otherwise, it means that there is a shorter path and it must contain another unvisited node.
Let $v'$ be the first unvisited node on this path, then $T(v') < T(v)$, which
contradicts the fact that $T(v)$ is minimum for all unvisited nodes. Therefore we know that 
$T(v)$ must be the shortest time needed from $s$ to $v$.

Hence the hypothesis also holds for $n$ visited nodes. By induction we know that the hypothesis is
true.

\Complexity
The proposed algorithm has same time complexity as Dijkstra's algorithm, which is $O(|E| +
|V|\log|V|)$, where $|E|$ is the number of edges, and $|V|$ is the number of vertices.

\section*{Problem 3: Timing circuits}
\Algorithm
For each node $v$, we use $L(v)$ to represent the largest root-to-leaf distance from $v$ to 
all leaves that are descendants of $v$.

Starting from root node $v$, let $v_{l}$ and $v_{r}$ represent its left and right child. We do the
following:
\begin{enumerate}
  \item If $L(v_l) > L(v_r)$, we add $L(v_l) - L(v_r)$ to the length of edge between $v$ and $v_r$.
  \item If $L(v_l) < L(v_r)$, we add $L(v_r) - L(v_l)$ to the length of edge between $v$ and $v_l$.
  \item If $L(v_l) = L(v_r)$, we do nothing.
\end{enumerate}
We do the same recursively to the left and right subtress of $v$ until we go to the leaf node.

\Proof
\textbf{Claim 1}: for any non-leaf node $v$, and let $v_r$ and $v_l$ be two child of $v$. If a
solution adds
length to both edges $e_r = (v, v_r)$ and $e_l = (v, v_l)$, the solution is not optimal.

\textbf{Proof}: Suppose we have added length $l_r, l_l$ to $e_r, e_l$ respectively. Without lose of
generality, we assume $l_l < l_r$. Obviously we can simply add $l_r - l_l$ to $e_r$, and after this
the tree still has zero skew. Since $l_r - l_l < l_r + l_l$, the original solution is not optimal.

\textbf{Claim 2}: let $v$ be an non-root internal node, if a solution increases the length of all paths from
$v$ to the leaves in the subtree rooted at $v$, then this solution is not optimal.

\textbf{Proof}: since the lengths of paths from $v$ to all leaves has been increased, for each
of such paths, it must include an edge $e$ whose length has been increased. For each path, we select
the first of such edges on the path, and use $Q$ to represent this set. We know that the number of edges
in $Q$ must be more than $2$ because paths from $v$ to leaves on left subtree and paths from $v$ 
to leaves on right subtree cannot share a common edge. Given this, let $l$ be the minimum increase
of length for edges in $Q$, we could reduce the  lengths of edges in $Q$ by $l$, and add $l$ to the
edge between $v$ and its parent. After this, the tree still has zero skew. In this process, we have
reduced a length of $|Q|l$ and increased a length of $l$. Since $|Q|l \geq 2l > l$, the modified solution increase
the length less than original solution. Hence the original solution is not optimal.

Given above claims, we can prove that the proposed algorithm is optimal. Otherwise, assume that
at an node the solution doesn't work same as the proposed algorithm. Let $v$ be such a node, and
$v_l, v_r$ be its left and right child respectively. Suppose that the solution adds $l_l, l_r$ to 
the edge $e_l = (v, v_l)$ and $e_r = (v, v_r)$. Without lose of generality, we assume that $l_l <
l_r$. 

There are following three cases:
\begin{enumerate}
  \item $l_l  + L(v_l) = l_r + L(V_r)$, and $l_l \neq 0$. In this case, by Claim 1 we know that this
    solution is not optimal. Note that $l_l$ cannot be zero, otherwise it is the same as proposed
    algorithm.

  \item $l_l  + L(v_l) > l_r + L(V_r)$. In this case, we know that the solution has to increase the
    length of paths from $v_r$ to all leaves in the subtree rooted at $v_r$. By Claim 2, we know
    that this solution is not optimal.

  \item $l_l  + L(v_l) < l_r + L(V_r)$. Similarly, we know that the solution has to increase the
    length of paths from $v_l$ to all leaves in the subtree rooted at $v_l$. By Claim 2, we know
    that this solution is not optimal.
\end{enumerate}
Therefore we know that the solution is not optimal. Hence the proposed algorithm is optimal.

\Complexity
To find the maximal node-to-leaf distance for each node , we need to traverse the complete binary tree, 
which has time complexity of $O(|V|)$. To adjust the length of edges, each node is also visited
once,  which also has time complexity of $O(|V|)$. Therefore, the time complexity of the proposed
algorithm is $O(|V|)$.

\section*{Problem 4: Oxen pairing}
\Algorithm
For all oxen in the group, we first pick the strongest ox $p$ and the weakest ox $q$ ($p$ and $q$
must be different), if they 
are able to pull a plow, that is $S_p + S_q \geq P$, then we pair them as a team and delete them
from the group, otherwise we delete the weakest ox from the group. Repeat this procedure until there 
are less than two oxen remaining in the group. Return the number of teams formed as the maximal
number of teams.

\Proof
\textbf{Claim 1}: 
Let $p$ and $q$ be the strongest and weakest ox in the group, if $S_p + S_q < P$,
then $q$ will not be included in any team.

Proof: If $S_p + S_q < P$, then for the weakest ox $q$, for any other ox $x$, we have $S_x + S_q <
S_p + S_q < P$, that is, $q$ cannot be paired with any ox to form a team to pull the plow. Claim 1
is proved.

\vspace{0.3cm}
\textbf{Claim 2}: 
Let $p$ and $q$ be the strongest and weakest ox in the group, if $S_p + S_q \geq P$, let $T_1$ be a
set of teams that can pull the plow and doesn't include pair $(p, q)$, we claim that there exists
$T_2$ which is another set of teams that include pair $(p, q)$ and $|T_2| \geq |T_1|$.

Proof: first if neither $p$ or $q$ is in the teams of $T_1$, then obviously we can add team $(p,
q)$ to team $T_1$ to form $T_2$, and $|T_2| = |T_1| + 1 > |T_1|$.

If $p$ belongs to a team, $q$ doesn't belong to a team in $T_1$, let $(p, x)$ be the team, since 
$S_p + S_q > P$, then obviously we use team $(p, q)$ to replace team $(p, x)$ to form $T_2$, and 
$|T_2| = |T_1|$.


If $q$ belongs to a team, $p$ doesn't belong to a team in $T_1$, similarly, let $(q, x)$ be the team, since 
$S_p + S_q > P$, then obviously we use team $(p, q)$ to replace team $(q, x)$ to form $T_2$, and 
$|T_2| = |T_1|$.

If both $p$ and $q$ belong to teams in $T_1$ and $p, q$ are not in a team, let $(p, x)$, $(q, y)$ be
the teams in $T_1$. We have the following:
\begin{align}
  S_x + S_y \geq S_q + S_y \geq P 
\end{align}
that is we can use $(x, y), (p, q)$ to replace $(p, x), (q, y)$ in $T_1$ to form $T_2$. And $|T_2| =
|T_1|$.
Given above, Claim 2 can be proved.

\vspace{0.3cm}
Following we use induction to prove that the result generate by proposed algorithm is optimal.
For base case, obviously the proposed algorithm is correct for $N <= 2$.  
Assume that the proposed algorithm is optimal when number of oxen is smaller than $N$. Then when
there are $N$ oxen in the group, let $p$ and $q$ be the strongest and weakest ox.

If $S_p + S_q < P$, according to Claim 1, we know that $q$ cannot form team with any other ox. We
can safely remove $q$ from the group, then the problem reduces to the case of $N-1$. By induction,
the proposed algorithm will generate optimal result.

If $S_p + S_q \geq P$, for any solution $T_1$, according to Claim 2, we have a solution $T_2$ that
includes pair $(p, q)$ and $|T_2| \geq |T_1$. Let $T^*$ be the solution generated by proposed
algorithm, then $T*$ must include $(p, q)$, and by induction $T^* - (p, q)$ is an optimal solution for the group
of $N-2$ oxen after removing $p$ and $q$, and $T_2 - (p, q)$ is a valid solution to it. Therefore we
have the following:
\begin{align}
  |T_2| - 1 & \leq |T^*| - 1 \\
  |T_2| & \leq |T^*|
\end{align}
so we have $|T_1| \leq |T_2| \leq |T^*|$. That is $T^*$ is also optimal.

By induction, we know that the proposed algorithm is optimal for all $N$.

\Complexity
An implementation of the proposed algorithm is as following:
\begin{minted}[frame=lines, framesep=2mm]{cpp}
  Input: [Ox_1, Ox_2,.., Ox_n]
  Output: T
  
  // Sort all oxen in ascending order of strength
  [Ox_1, Ox_2,.., Ox_n] = sort([Ox_1,.., Ox_n]);
  i = 1;
  j = n;
  while (i < j) {
    if (S(Ox_i) + S(Ox_j) >= P) {
      add (Ox_i, Ox_j) to T;
      i++;
      j--;
    } else {
      i++;
    }
  }
  return T;
\end{minted}

The time complexity of sorting all oxen is $O(n\log n)$. The loop is executed for at most $n-1$
times, which has time complexity of $O(n)$. Therefore, the time complexity of the proposed algorithm
is $O(n\log n)$.

\section*{Problem 5}
\subsection*{Part 1}
\Algorithm
We construct a new graph $G^*$ with the set of nodes $V$. In the new graph, for two nodes $p$ and $q$,
we add an edge between them if and only if there is an edge between them in all graphs $G_0, G_1,
\dots, G_b$.

With this graph, we can find the shortest path between $s$ and $t$ with \textit{Breadth-first
search(BFS)} 
algorithm.

\Proof
Let $L$ be the target $s-t$ path. First we show that such a path $L$ must exist in $G^*$. Since
every edge on $L$ is connected on all graphs, so every edge on $L$ must exist in $G^*$. Given this,
obviously $L$ exists in $G^*$. Since \textit{BFS} algorithm always returns the shortest
path between $s$ and $t$, so the returned path by proposed algorithm must be the shortest path
between $s$ and $t$. Therefore the proposed algorithm is correct.

\Complexity
To construct the graph, we need to find the common edges among $E_0,E_1,\dots,E_b$, the complexity
of this step is $O(\sum_{i=0}^{b}|E_i|)$. Also we need to add all the nodes, and the time complexity 
is $O(|V|)$.

To find the shortest path between $s$ and $t$, the time complexity of $E$ is $O(\min\{|E_i|\})$.

In summary, the time complexity of the proposed algorithm is $O(\sum_{i=0}^{b}|E_i|)$.

\subsection*{Part 2}
\Algorithm
We apply dynamic programming to solve this problem. Use $S(i)$ to represent to cost of optimal solution
for $P_0, P_1, \dots, P_i$. And also we use $L_{i, j}$ to represent the length of $s-t$ path
in all graphs $G_i, G_{i+1}, \dots, G_j$. If no such $s-t$ path exists, we set $L_{i, j}$ to
$+\infty$.

To calculate $S(b)$, suppose that the last index $i$ where $P_i \neq P_{i+1}$. The cost of this
solution is as following:
\begin{align}
  cost = S(i) + K + (b - i)L(i+1, b)
\end{align}
Therefore, $S(b)$ can be derived as following:
\begin{align}
  S(b) = \min_{0\leq i < b} S(i) + K + (b-i)L(i+1, b)
\end{align}
Here we don't include the case where $P_i = P_{i+1}$ for $ 0 \leq i < b$. In this case, we have 
$S(b) = (b+1)L(0, b)$. Therefore, the complete solution should be:
\begin{align}
  S(b) = \min((b+1)L(0,b), \min_{0\leq i < b} S(i) + K + (b-i)L(i+1, b))
\end{align}


\section*{Problem 6}


\end{document}




















%%%%%%%%%%%%%%%%%%%%%%%%%%%%%%%%%%%%%%%%%%%%%%%%%%%%%%%%%%%%%
