\documentclass{article}
\usepackage{amsmath,amsfonts,amsthm,amssymb}
\usepackage{setspace}
\usepackage{fancyhdr}
\usepackage{lastpage}
\usepackage{extramarks}
\usepackage{chngpage}
\usepackage{soul,color}
\usepackage{graphicx,float,wrapfig}
\usepackage{clrscode}
\usepackage{mathrsfs}
\newcommand{\Class}{CSE 202}
% \newcommand{\ClassInstructor}{Russell Impagliazzo}

% Homework Specific Information. Change it to your own
\newcommand{\Title}{Homework 3}
\newcommand{\StudentName}{Sai Bi}
\newcommand{\StudentClass}{}
\newcommand{\StudentNumber}{}

% In case you need to adjust margins:
\topmargin=-0.45in      %
\evensidemargin=0in     %
\oddsidemargin=0in      %
\textwidth=6.5in        %
\textheight=9.0in       %
\headsep=0.25in         %

% Setup the header and footer
\pagestyle{fancy}                                                       %
\lhead{\Title}  %
\rhead{\firstxmark}                                                     %
\lfoot{\lastxmark}                                                      %
\cfoot{}                                                                %
\rfoot{Page\ \thepage\ of\ \protect\pageref{LastPage}}                          %
\renewcommand\headrulewidth{0.4pt}                                      %
\renewcommand\footrulewidth{0.4pt}                                      %

%%%%%%%%%%%%%%%%%%%%%%%%%%%%%%%%%%%%%%%%%%%%%%%%%%%%%%%%%%%%%
% Some tools
\newcommand{\enterProblemHeader}[1]{\nobreak\extramarks{#1}{#1 continued on next page\ldots}\nobreak%
    \nobreak\extramarks{#1 (continued)}{#1 continued on next page\ldots}\nobreak}%
\newcommand{\exitProblemHeader}[1]{\nobreak\extramarks{#1 (continued)}{#1 continued on next page\ldots}\nobreak%
    \nobreak\extramarks{#1}{}\nobreak}%

\providecommand{\myceil}[1]{\left \lceil #1 \right \rceil }
\newcommand{\homeworkProblemName}{}%
\newcounter{homeworkProblemCounter}%
\newenvironment{homeworkProblem}[1][Problem \arabic{homeworkProblemCounter}]%
{\stepcounter{homeworkProblemCounter}%
    \renewcommand{\homeworkProblemName}{#1}%
    \section*{\homeworkProblemName}%
    \enterProblemHeader{\homeworkProblemName}}%
{\exitProblemHeader{\homeworkProblemName}}%

\newcommand{\homeworkSectionName}{}%
\newlength{\homeworkSectionLabelLength}{}%
\newenvironment{homeworkSection}[1]%
{% We put this space here to make sure we're not connected to the above.
    
    \renewcommand{\homeworkSectionName}{#1}%
    \settowidth{\homeworkSectionLabelLength}{\homeworkSectionName}%
    \addtolength{\homeworkSectionLabelLength}{0.25in}%
    \changetext{}{-\homeworkSectionLabelLength}{}{}{}%
    \subsection*{\homeworkSectionName}%
    \enterProblemHeader{\homeworkProblemName\ [\homeworkSectionName]}}%
{\enterProblemHeader{\homeworkProblemName}%
    
    % We put the blank space above in order to make sure this margin
    % change doesn't happen too soon.
    \changetext{}{+\homeworkSectionLabelLength}{}{}{}}%

\newcommand{\Answer}{\ \\\textbf{Answer:} }
\newcommand{\Acknowledgement}[1]{\ \\{\bf Acknowledgement:} #1}
\newcommand{\Complexity}{\vspace{0.3cm} \noindent\textbf{Time Complexity} \\}
\newcommand{\Proof}{\vspace{0.3cm} \noindent\textbf{Proof} \\}
\newcommand{\Algorithm}{\textbf{Algorithm} \\}
\newcommand{\EndProof} { \hfill$\square$ }
\newcommand\equ[1]{\begin{align}\begin{split} #1 \end{split} \end{align}}
%%%%%%%%%%%%%%%%%%%%%%%%%%%%%%%%%%%%%%%%%%%%%%%%%%%%%%%%%%%%%
\setlength\parindent{0pt}
\setlength{\parskip}{0.2cm}%% 
\usepackage{minted}
%%%%%%%%%%%%%%%%%%%%%%%%%%%%%%%%%%%%%%%%%%%%%%%%%%%%%%%%%%%%%
% Make title
\title{\textmd{\bf \Class: \Title}}
\date{}
\author{\textbf{\StudentName}}
%%%%%%%%%%%%%%%%%%%%%%%%%%%%%%%%%%%%%%%%%%%%%%%%%%%%%%%%%%%%%

\begin{document}
\maketitle \thispagestyle{empty}
\section*{Problem 1: Job scheduling}
\Algorithm
We solve this problem by network flow. For each job $j$, we create a node $v_j$. For each time unit
$t_p$ during which at least one of the processor is available, we add a node $u_p$. Also we add a
directed edge from source $s$ to $v_j$ and from $u_p$ to sink $t$. We add other edges and set their
capacities as following:
\begin{enumerate}
  \item The capacity of edge from start node $s$ to job $v_j$ is equal to the processing time
    $v_j$ needs, that is $C(s, v_j) = \ell_j$. 
  \item The capacity of edge from node $u_p$ to sink $t$ is equal to the number of processors that
    is available at this time unit. That is $C(u_p, t) = |\{i | t_i \leq u_p < t_i', 1 \leq i \leq k\}|$.  
  \item For each node $v_j$ and $u_p$, if $a_j \leq u_p$ and $d_j > u_p$, we add an edge between $v_j$
    and $u_p$ with a capacity of $1$. In this way, we can guarantee that the job cannot be assigned
    to a processor before it arrives and after the deadline passes. Also by setting capacity to $1$
    we guarantee that a job can only be set to one processor at a time unit.  
\end{enumerate}

Given above formulation, we can find the maximum flow of the network. If the capacity of the maximum
flow is equal to the sum of lengths of all jobs $\sum_{j} \ell_j$, then we know that the jobs can
all be completed. Otherwise, the jobs cannot be completed. To determine a proper scheduling when the
jobs can all be completed, suppose there is an edge with capacity $1$ between $v_j$ and $u_p$, we
know at time $t_p$ that job $v_j$ can be assigned to a remaining processor that is available at $t_p$, and delete it from
available list at $t_p$.

\Proof
Let us see that an 

\section*{Problem 2: Graph cohesiveness}

\section*{Problem 3: Number puzzle}

\section*{Problem 4: Database projections}

\section*{Problem 5: Maximum likelihood points of failure}

\end{document}

