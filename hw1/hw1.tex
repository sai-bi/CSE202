\documentclass[a4paper,11pt]{article}

\usepackage[T1]{fontenc}
\usepackage[utf8]{inputenc}
\usepackage{graphicx}
\usepackage{xcolor}
 \usepackage{tgtermes}

 \usepackage[
 pdftitle={Math Assignment},
 pdfauthor={Joe Doe, Some University},
 colorlinks=true,linkcolor=blue,urlcolor=blue,citecolor=blue,bookmarks=true,
 bookmarksopenlevel=2]{hyperref}
\usepackage{amsmath,amssymb,amsthm,textcomp}
\usepackage{enumerate}
\usepackage{multicol}
\usepackage{tikz}

\usepackage{geometry}
\geometry{total={210mm,297mm},
left=25mm,right=25mm,%
bindingoffset=0mm, top=20mm,bottom=20mm}


\linespread{1.3}

\newcommand{\linia}{\rule{\linewidth}{0.5pt}}

\providecommand{\myceil}[1]{\left \lceil #1 \right \rceil }
% custom theorems if needed
\newtheoremstyle{mytheor}
    {1ex}{1ex}{\normalfont}{0pt}{\scshape}{.}{1ex}
    {{\thmname{#1 }}{\thmnumber{#2}}{\thmnote{ (#3)}}}

\theoremstyle{mytheor}
\newtheorem{defi}{Definition}
\usepackage[ruled, vlined, linesnumbered,lined,boxed,commentsnumbered]{algorithm2e}
\usepackage[parfill]{parskip}
% my own titles
\makeatletter
\setlength\parindent{0pt}
% custom footers and headers
\usepackage{fancyhdr,lastpage}
\begin{document}

\title{CSE 202: Design and Analysis of Algorithms}

\author{Sai Bi}

\date{\today}

\maketitle

\section*{Problem 1}
Let $N$ be the length of the array $A$. The algorithm works as following:
\begin{enumerate}
\item
Do a pass to the array and get the array $B$ where $B[i]=\max\limits_{1\leq k \leq i} A[k]$.
\item 
Do a pass to the array and get the array $C$ where $C[i]=\min\limits_{i\leq k \leq N} A[k]$.
\item
Return $\max\limits_{1\leq k \leq N-1} (B[i] - C[i+1])$.
\end{enumerate}
The time complexity of the algorithm is $O(N)$.

\section*{Problem 2}
The algorithm works as following: do a scan from left to right to the string, every time we find
a unmatched right bracket, we change it into a left bracket. Assume that we have $2n$ unmatched 
brackets remaining, we change the rightmost $n$ left brackets to right brackets. Then the string
will be balanced.


\section*{Problem 3}
\end{document}
